\chapter{Introdução}
\label{cap-introducao}

A Engenharia de Software tem evoluídos seus métodos e técnicas para prover melhorias no desenvolvimento de software com objetivos baseados em cumprimento de prazos e orçamentos assim como a implementação de produtos que atendem parâmetros de qualidades desejados. Estas melhorias são observáveis em diferentes pontos, desde o processo ao produto, cujos objetivos e prioridades podem variar de acordo com a metodologia de desenvolvimento. Apesar de suas diferenças conceituais e de valores, a maior parte dos métodos preveêm processos e técnicas referentes ao design, testes e medição, que visam garantir a qualidade do software em desenvolvimento.

%

No contexto de projetos que adotam metodologias ágeis observa-se que tanto a qualidade interna quanto a qualidade externa do software são preponderantes, pois são fatores fundamentais para suportar a simplicidade, o feedback contínuo e adaptação à mudanças, valores que solidificam o desenvolvimento ágil. A qualidade interna do software é observada a partir de atributos de qualidades na perspectiva de desenvolvimento que, segundo Berander (\citeyear{berander2005}), se resumem em corretude, testabilidade, flexibilidade, portabilidade, reusabilidade, interoperabilidade, analisabilidade, adaptatividade e estabilidade. As práticas ressaltadas pela metologia Extreme Programming \cite{beck2000} visam realçar os valores dos atributos destacados. O design simples pode ser obtido através de técnicas como o Desenvolvimento Orientado à Testes (CITAR UMA REFERÊNCIA) e o Refactoring (CITAR OUTRA REFERÊNCIA), que por sua vez influenciam diretamente os atributos testabilidade, reusabilidade e adaptatividade. Ambas as técnicas se baseiam fortemente em testes unitários (CITAR OUTRA REFERÊNCIA) que provê a segurança necessária para realização de mudanças assim como o feedback automatizado da manutenção do software. O Pair Programming (CITAR REFERÊNCIA), dentre outras técnicas, também possui papel fundamental na garantia da qualidade interna, uma vez que exercita a programação e revisão ao mesmo tempo, reduzindo a ocorrência de não-conformidades técnicas e inserção de bugs. Por outro lado, a qualidade externa do software pode ser alcançada a partir do envolvimento do cliente ao longo das atividades de desenvolvimento e, principalmente, a partir de entregas contínuas de software com valor de negócio.

%

Valores semelhantes podem ser observados nas comunidades de desenvolvimento de softwares livres refletindo diretamente na alta qualidade do código produzido em diversos projetos livres \cite{schmidt2001}; \cite{halloran2002}; \cite{michlmayr2003}. Essas comunidades adotam a padronização de código e testes automatizados para manter a qualidade interna do código e incentivar a contribuição de diversos desenvolvedores.

A melhoria da qualidade interna do código apoia a melhoria contínua do processo oferecendo subsídios para que a equipe de desenvolvimento aumente sua produtividade e implemente novas funcionalidades com maior facilidade. Beck (\citeyear{beck2007}) corrobora esta afirmação ao destacar que a maior parte do tempo utilizado por um Programador ao inserir novas funcionalidades é destinado ao entendimento do código em manutenção. 
% Falar sobre a segurança de software vs qualidade 

Diretamente relacionado a qualidade de código está a sua segurança (MAC GRAW et al  - SEVEN KING..). A segurança de software está relacionada a confiabilidade, disponibilidade e integridade dos diversos componentes que compõe o software. Dados do ICAT/NIST de 2005 já apontavam que 80\% das vulnerabilidades remotamente exploráveis estavam ligadas a má codificação do programa. Embora a segurança de uma aplicação também estejam relacionadas a aspectos externos ao software como a redes e componentes de hardware, o elo mais fraco continua sendo o próprio software. Dessa forma, cabe aos projetistas e desenvolvedores a responsabilidade do desenvolvimento de software seguro, sem prejuízos aos seus usuários.

%http://cve.mitre.org/docs/docs-2002/prog-rpt_06-02/ CITAR
A medida que o tempo vai passando, novas vunerabilidades vão sendo descobertas pela comunidade. O projeto CVE, que tem como objetivo enumerar vunerabilidades de software existentes, tinha uma lista de 321 vunerabilidades diferentes no ano de sua concepção, 1999. No ano de 2002, a lista já havia aumentado para 2032 vunerabilidades e atualmente o número já chega a 61 mil vunerabilidades específicas encontradas por empresas de todo o mundo.

Visto esse cenário de inúmeras vulnerabilidades é fundamental que todos aqueles envolvidos no processo de produção do software tenham conhecimento das implicações relativas a segurança. O conhecimento de vunerabilidades e meios de detecta-las são habilidades necessárias para garantia de software seguro.

%

Neste sentido, a medição pode ser utilizada como um processo de apoio ao acompanhamento da segurança e qualidade, através do estabelecimento de metas e indicadores que indiquem oportunidades de melhorias observáveis do produto. Em um cenário otimista, os próprios Engenheiros de Software podem adotar como prática a medição do código-fonte para auxiliar as tomadas de decisões, ou até mesmo para avaliação do código inserido ou da aplicação de refactoring.

%falar sobre ferramentas de monitoramento e DW

Porém, a grande quantidade de métricas, coletas manuais e poucos recursos de visualização são fatores que acabam por desmotivar o uso destas para o monitoramento do código. Além de que compreender o significado de valores que uma métrica isolada possui não é uma tarefa fácil.

E como estamos falando do uso de métricas para a tomada de decisão, uma solução que se tem destaque no ramo de BI e tem enfase em fornecer uma ambiente de fácil acesso a informação para a tomada de decisão é um ambiente de Data Warehousing. O Data Warehouse, sendo um componente desse ambiente, constitui-se de uma base de dados que procura de maneira eficiente e flexível tratar de grande volume de dados e obter informações que auxiliem no processo de tomada de decisão (OLIVEIRA , 2007).

No contexto de monitoramento de métricas de software, o uso de DWing para esta finalidade tem se mostrado como boa solução, como pode ser visto nos trabalhos de (CHULANI et al.,2003) (CASTELLANOS et al., 2005) (PALZA; FUHRMAN; ABRAN, 2003) (FOLLECOet al., 2007) (SILVEIRA, 2007). 


Portanto, neste trabalho serão exploradas a utilização de métricas para o monitoramento de código-fonte para compreender e estabelecer possíveis relações existentes entre as mesmas no que diz respeito a vunerabilidades de software. Assim, espera-se identificar as oportunidades de utilização de métricas na melhoria contínua do processo e, consequentemente, na qualidade interna do produto a partir do estabelecimento de cenários, compostos a partir da análise de correlação de métricas, que evidenciem as boas e más características do design de um projeto que impactam na vunerabilidade do sistema com o objetivo de facilitar a interpretação e evitar possíveis equívocos que são baseados em análises errôneas sobre métricas isoladas, sobre correlações inexistentes ou até mesmo a escolha de métricas inadequadas cujos problemas são discutidos em \cite{chidamber1994}. Para auxiliar no monitoramento e na tomada de decisão, será explorado o uso de plataforma de monitoramento baseado em métricas e um ambiente de DWing e será observado o uso dessas duas soluções e suas contribuições para a melhoria do processo e qualidade do produto..

%------------------------------------------------------------------------------%

\section{Objetivos}

%O objetivo deste trabalho permeia um estudo teórico de conceitos relacionados a métricas de monitoramento de código-fonte, buscando identificar correlações existentes entre algumas métricas e suas interpretações que evidenciem tanto características de bom design, tais como Código Limpo \cite{almeida1994}, quanto bad smells (CITAR REFERÊNCIA). O estabelecimento destes cenários permite a seleção de métricas adequadas baseadas em determinados objetivos de melhorias, facilta a interpretação tanto a nível gerencial quanto a nível de desenvolvimento e apoia a melhoria contínua de software a partir da adoção do monitoramento de código-fonte como uma prática de desenvolvimento.

O objetivo deste trabalho consite no estudo teórico sobre conceitos relacionados a métricas de monitoramento de código-fonte buscando relacionar características de bom design, como Código limpo, quanto Bad Smells no contexto de vunerabilidades de software  para o estabelecimento de cenários que caracterizam determinado componente do sistema e auxiliem na tomada de decisão no que diz respeito a refatoração de código para melhoria da segurança da aplicação.

%

Além disso, tem-se como objetivo nesta monografia oferecer duas propostas de ambientes de visualização e monitoramento do código com base nos cenários definidos:
\begin{itemize}
\item \textbf{Plataforma livre de monitoramento de métricas}: Evolução da plataforma Mezuro para suportar a criação de configurações que caracterizem os cenários estabelecidos;
\item \textbf{Ambiente de DWing}: Criação de um ambiente de DWing para extração e análise das métricas baseada nos cenários definidos para a ajuda na tomada de decisão;
\end{itemize}

Assim, pretende-se evidenciar como estes cenários podem ser utilizados através da ferramenta de análise estática de código automatizada incorporados as boas práticas de desenvolvimento das Engenharia de Software.
%% OU



%------------------------------------------------------------------------------%

\section{Metodologia e Pesquisa}
