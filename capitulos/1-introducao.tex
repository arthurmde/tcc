\chapter{Introdução}
\label{cap-introducao}

A Engenharia de Software tem evoluído seus métodos e técnicas para prover melhorias no desenvolvimento de software com objetivos baseados em cumprimento de prazos e orçamentos assim como a implementação de produtos que atendem parâmetros de qualidades desejados. Estas melhorias são observáveis em diferentes pontos, desde o processo ao produto, cujos objetivos e prioridades podem variar de acordo com a metodologia de desenvolvimento. Apesar de suas diferenças conceituais e de valores, a maior parte dos métodos preveêm processos e técnicas referentes ao \emph{design}, testes e medição, que visam garantir a qualidade do software em desenvolvimento.

%

No contexto de projetos que adotam metodologias ágeis observa-se que tanto a qualidade interna quanto a qualidade externa do software são preponderantes, pois são fatores fundamentais para suportar a simplicidade, o feedback contínuo e adaptação à mudanças, valores que solidificam o desenvolvimento ágil. A qualidade interna do software é observada a partir de atributos de qualidades na perspectiva de desenvolvimento que, segundo Berander (\citeyear{berander2005}), se resumem em corretude, testabilidade, flexibilidade, portabilidade, reusabilidade, interoperabilidade, analisabilidade, adaptatividade e estabilidade. As práticas ressaltadas pela metologia \emph{Extreme Programming} \cite{beck2000} visam realçar os valores dos atributos destacados. O \emph{design} simples pode ser obtido através de técnicas como o Desenvolvimento Orientado à Testes \cite{beck2002} e a Refatoração \cite{fowler1999}, que por sua vez influenciam diretamente os atributos testabilidade, reusabilidade e adaptatividade. Ambas as técnicas se baseiam fortemente em testes unitários que proveem a segurança necessária para realização de mudanças assim como o feedback automatizado da manutenção do software. A Programação em Pares, dentre outras técnicas, também possui papel fundamental na garantia da qualidade interna, uma vez que exercita a programação e revisão ao mesmo tempo, reduzindo a ocorrência de não-conformidades técnicas e inserção de bugs. Por outro lado, a qualidade externa do software pode ser alcançada a partir do envolvimento do cliente ao longo das atividades de desenvolvimento e, principalmente, a partir de entregas contínuas de software com valor de negócio.

%

Valores semelhantes podem ser observados nas comunidades de desenvolvimento de softwares livres refletindo diretamente na alta qualidade do código produzido em diversos projetos livres \cite{schmidt2001}; \cite{halloran2002}; \cite{michlmayr2003}. Essas comunidades adotam a padronização de código e testes automatizados para manter a qualidade interna do código e incentivar a contribuição de diversos desenvolvedores.

%

A melhoria da qualidade interna do código apoia a melhoria contínua do processo oferecendo subsídios para que a equipe de desenvolvimento aumente sua produtividade e implemente novas funcionalidades com maior facilidade. Beck (\citeyear{beck2007}) corrobora esta afirmação ao destacar que a maior parte do tempo utilizado por um Programador ao inserir novas funcionalidades é destinado ao entendimento do código em manutenção. 
% Falar sobre a segurança de software vs qualidade 

%

Diretamente relacionado a qualidade de código está a sua segurança\cite{tsipenyuk2005}. A segurança de software está relacionada a confiabilidade, disponibilidade e integridade dos diversos componentes que compõe o software. Dados do ICAT/NIST de 2005 já apontavam que 80\% das vulnerabilidades remotamente exploráveis estavam ligadas a má codificação do programa \cite{duarte2005}. Embora a segurança de uma aplicação também estejam relacionadas a aspectos externos ao software como a redes e componentes de hardware, o elo mais fraco continua sendo o próprio software. Dessa forma, cabe aos projetistas e desenvolvedores a responsabilidade do desenvolvimento de software seguro, sem prejuízos aos seus usuários.

%

%http://cve.mitre.org/docs/docs-2002/prog-rpt_06-02/ CITAR
A medida que o tempo vai passando, novas vunerabilidades vão sendo descobertas pela comunidade. O projeto CVE, que tem como objetivo enumerar vunerabilidades de software existentes, tinha uma lista de 321 vunerabilidades diferentes no ano de sua concepção, 1999 \cite{cve2002}. No ano de 2002, a lista já havia aumentado para 2032 vunerabilidades e atualmente o número já chega a 61 mil vunerabilidades específicas encontradas por empresas de todo o mundo.

%

Visto esse cenário de inúmeras vulnerabilidades é fundamental que todos aqueles envolvidos no processo de produção do software tenham conhecimento das implicações relativas a segurança. O conhecimento de vunerabilidades e meios de detectá-las são habilidades necessárias para garantia de software seguro. Este conhecimento deve estar alinhado às habilidades de concepção de um bom \emph{design} para prover a segurança necessária no desenvolvimento do software e se alcançar os valores e objetivos anteriormente destacados.

%

Neste sentido, a medição pode ser utilizada como um processo de apoio ao acompanhamento da segurança e qualidade, através do estabelecimento de metas e indicadores que indiquem oportunidades de melhorias observáveis do produto. Em um cenário otimista, os próprios Engenheiros de Software podem adotar como prática a medição do código-fonte para auxiliar as tomadas de decisões, ou até mesmo para avaliação do código inserido ou da aplicação de refatorações.

%falar sobre ferramentas de monitoramento e DW

Porém, a grande quantidade de métricas, coletas manuais e poucos recursos de visualização são fatores que acabam por desmotivar o uso destas para o monitoramento do código. Além disso, a compreenção do significado de valores obtidos através de métricas não é uma tarefa trivial, demandando um grande esforço de interpretação necessárias para a tomada de decisão efetiva sobre o projeto de software.

%

Assim, destaca-se a importância de ferramentas que auxiliem o processo de medição, compreensão e visualização do software. Atualmente existem algumas ferramentas que automatizam a extração de métricas do código-fonte com objetivo de coletar as informações sobre o produto a partir da análise estática do código, as quais definimos como Extratores. Outras ferramentas denominadas Plataformas de Monitoramento de Código-Fonte procuram oferecer melhores formas de monitoramento e visualização do software a partir da personalização de métricas e mecanismos que faciltem a interpretação dos resultados obitidos. Alternativamente, o ambiente de \emph{Data Warehousing} (DWing) é uma solução que tem se destacado no ramo de BI e tem enfase em fornecer uma ambiente de fácil acesso a informação para a tomada de decisão. O Data Warehouse constitui-se de uma base de dados que procura de maneira eficiente e flexível tratar de grande volume de dados e obter informações que auxiliem no processo de tomada de decisão \cite{lopes2007}. Alguns trabalhos já têm utilizados o DWing no contexto de monitoramento de métricas apresentando bons resultados  \cite{Castellanos2005} \cite{Folleco2007} \cite{Silveira2010}\cite{mazuco2011}.

%

Portanto, neste trabalho serão exploradas a utilização de métricas para o monitoramento de código-fonte para compreender e estabelecer possíveis relações existentes entre as mesmas no que diz respeito a vunerabilidades e qualidade de software. Assim, espera-se identificar as oportunidades de utilização de métricas na melhoria contínua do processo e, consequentemente, na qualidade interna do produto a partir do estabelecimento de cenários, compostos a partir da análise de correlação de métricas, que evidenciem as boas e más características do \emph{design} de um projeto que impactam na vunerabilidade do sistema com o objetivo de facilitar a interpretação e evitar possíveis equívocos que são baseados em análises errôneas sobre métricas isoladas, sobre correlações inexistentes ou até mesmo a escolha de métricas inadequadas cujos problemas são discutidos em \cite{chidamber1994}. Para auxiliar no monitoramento e na tomada de decisão, será explorado o uso de plataforma de monitoramento de código-fonte e um ambiente de DWing e será observado o uso dessas duas soluções e suas contribuições para a melhoria do processo e qualidade do produto.

%------------------------------------------------------------------------------%

\section{Objetivos}

%O objetivo deste trabalho permeia um estudo teórico de conceitos relacionados a métricas de monitoramento de código-fonte, buscando identificar correlações existentes entre algumas métricas e suas interpretações que evidenciem tanto características de bom \emph{design}, tais como Código Limpo \cite{almeida1994}, quanto bad smells (CITAR REFERÊNCIA). O estabelecimento destes cenários permite a seleção de métricas adequadas baseadas em determinados objetivos de melhorias, facilta a interpretação tanto a nível gerencial quanto a nível de desenvolvimento e apoia a melhoria contínua de software a partir da adoção do monitoramento de código-fonte como uma prática de desenvolvimento.

O objetivo deste trabalho consite no estudo teórico sobre conceitos relacionados a métricas de monitoramento de código-fonte buscando relacionar características de bom \emph{design}, como Código Limpo e Bad Smells no contexto de vunerabilidades de software  para o estabelecimento de cenários que caracterizam determinado componente do sistema e auxiliem na tomada de decisão no que diz respeito à práticas de \emph{design} de código para melhoria da segurança da aplicação.

%

Além disso, tem-se como objetivo nesta monografia oferecer duas propostas de ambientes de visualização e monitoramento do código com base nos cenários definidos:
\begin{itemize}
\item \textbf{Plataforma livre de monitoramento de métricas}: Evolução da plataforma Mezuro para suportar a criação de configurações que caracterizem os cenários estabelecidos;
\item \textbf{Ambiente de DWing}: Criação de um ambiente de DWing para extração e análise das métricas baseada nos cenários definidos para a ajuda na tomada de decisão;
\end{itemize}

Assim, pretende-se evidenciar como estes cenários podem ser utilizados através da ferramenta de análise estática de código automatizada incorporados as boas práticas de desenvolvimento das Engenharia de Software. Uma comparação entre as duas propostas também cabem ao escopo deste trabalho.

%

\subsection{Questões de Pesquisa}

%

Este trabalho busca responder as seguintes questões de pesquisa:

%

\begin{itemize}
\item \textbf{QP1} - Um ambiente de \emph{Data Warehousing}(DWing) é  adequado para auxiliar na tomada de decisão tanto do ponto de vista estratégico gerencial quanto no ponto de vista de modificações de \emph{design} por desenvolvedores/Engenheiro de software?
\item \textbf{QP2} - O Mezuro é adequado para suportar a tomada de decisões gerenciais de projetos de software e para auxiliar o Engenheiro de Software na melhoria contínua do produto?
\item \textbf{QP3} - O Mezuro e o DWing podem ser utilizados em conjunto para auxiliar a tomada de decisões em vários níveis de projetos de software?
\item \textbf{QP4} - Métricas de \emph{design} de código possuem correlação com métricas de vulnerabilidades?
\item \textbf{QP5} - O monitoramento de vulnerabilidades é viável em projetos de softwares não- críticos?
\item \textbf{QP6} - Métricas estáticas podem ser compostas em cenários para a definição de indicadores mais informativos e completos?
\end{itemize}

%

Para responder estas questões de pesquisa, busca-se atingir os seguintes objetivos específicos:

%
\begin{description}
	\item [Objetivos Técnológicos]\
\end{description}
		\begin{itemize}
			\item \textbf{OT1} - Evoluir a plataforma livre Mezuro de monitoramento de código-fonte:
				\begin{itemize}
					\item Evoluir a arquitetura do Mezuro para melhorar sua modularização e flexibilidade.
					\item Evoluir a configuração de métricas do Mezuro
					\item Implementar mecanismos de definição de celos/cenários no Mezuro
					\item Melhorar mecanismos de visualização de software do Mezuro
				\end{itemize}
			\item \textbf{OT2} - Criar ambiente de Data Warehousing para monitoramento dos cenários de decisão no contexto de vulnerabilidade de software.	
	        	\begin{itemize}
	        		\item Implementar extração, transformação e carregamento da base de dados a partir de reports de ferramentas de análise estática na ferramenta Pentaho.
	        		\item Criar modelo dimensional
	        		\item Gerar Cubo de dados        
					\item Configurar mecanismos de visualização do cubo de dados na plataforma BI server.
	        	\end{itemize}
			\item \textbf{OT3} - Evoluir extração de métricas de vulnerabilidade para a ferramenta Analizo
		\end{itemize}
\begin{description}
	\item [Objetivos Científicos]\
\end{description}	
	 	\begin{itemize}
			\item \textbf{OC1} - Catálogar definições teóricas a respeito dos principais conceitos relacionados à vulnerabilidades de software.
			\item \textbf{OC2} - Estudo teórico sobre a relação de vulnerabilidades de software com o \emph{design}
			\item \textbf{OC3} - Análise estatísticas de dados de softwares livres a fim de comprovar correlação entre vunerabilidades e \emph{design} de software.
			\item \textbf{OC4} - Propor um modelo baseado em sequência de passos para o Engenheiro de Software no desenvolvimento de softwares robustos.
			\item \textbf{OC5} - Definição de cenários a partir de estudos teóricos para melhorar a interpretação e tomada de decisão sobre métricas estáticas de código-fonte.
	 	\end{itemize}





%------------------------------------------------------------------------------%

\section{Metodologia e Pesquisa}

%

Os objetivos científicos e tecnológicos deste trabalho contemplam passos para a alcançar o seu Objetivo Geral e responder às perguntas de pesquisa. Portanto, serão realizados estudos teóricos através de revisão bibliográfica para compreensão e definição dos conceitos básicos sobre características de \emph{design}  e de vulnerabilidades de software, contemplando os objetivos científicos OC1 e OC2.

%

Além das discussões obtidas a partir da revisão bibliográfica, para se alcançar o OC3 será projetado um expertimento para se avaliar a relação entre características de \emph{design} e vulnerabilidades de software. Este experimento consistirá em uma coleta de métricas em diversos projetos de software livre e a análise estatística de correlação entre essas métricas, onde pretende-se analisar softwares críticos e não críticos. Neste sentido, espera-se respoder as questões de pesquisa QP4 e QP5. Para tanto, será utilizado o modelo de regressão múltipla para verificar a correlação entre os dados. Iremos então determinar o impacto de métricas de \emph{design} (variáveis independentes) com a ocorrência de  vulnerabilidades específicas (varíaveis dependentes). Este experimento será projetado através de um protocolo de estudo experimental, contemplando a definição de hipóteses, seleção da amostra, desenho do estudo, estratégia de condução e análise de dados conforme proposto por Luna (\citeyear{luna1998}).

%

Para responder às questões de pesquisa Q1, Q2 e Q3, ou seja, verificar a adequação do ambiente de DWing e da plataforma de monitoramento Mezuro para auxílio na tomada de decisão orientada a métricas de software, será elaborado um protocolo de estudo de caso. Este protocolo será usado para guiar a condução do estudo de caso e deverá conter os procedimentos e regras que governam a condução da pesquisa \cite{miles1994}. O objetivo principal deste estudo de caso é avaliar aspectos e características das soluções frente às necessidades de projetos de software. Com o resultado da aplicação desse protocolo, também será possível comparar as duas soluções e verificar se ambas podem se complementar, fornecendo um serviço mais completa e com mais benefícios.

%

Por fim, baseado principalmente na revisão bibliográfica sobre \emph{design}, segurança e métricas estáticas de software, as duas ferramentas propostas, Mezuro e ambiente DWing, serão constrúidas e evoluídas. Esta atividade está relacionado aos objetivos tecnológicos deste trabalho de conclusão de curso, onde técnicas, métodos e padrões de Engenharia de Software poderão se aplicados.

%------------------------------------------------------------------------------%

\section{Organização do Trabalho}

Esta monografia está dividida em mais outros 4 capítulos. A seguir, o leitor encontrará o Capítulo \ref{cap-metrics} onde são estruturados e discutidos os principais conceitos relacionados a \emph{design}, vulnerabilidade e métricas de software. Posteriormente no Capítulo \ref{cap-mezuro} será apresentada a plataforma de monitoramento de código-fonte Mezuro, uma ferramenta livre que será evoluída de acordo com os Objetivos Tecnlógicos deste trabalho. O Capítulo \ref{cap-dw} apresenta os principais conceitos relacionados a Data Warehousing que também será explorada no contexto deste trabalho como solução para utilização de métricas de código-fonte. Por fim, no Capítulo \ref{cap-cenarios} serão apresentados o conceito de Cenários de Decisões e algumas propostas que podem ser utilizadas no monitoramento de projetos.