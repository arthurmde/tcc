\chapter{Introdução}
\label{cap-introducao}

A Engenharia de Software tem evoluído seus métodos e técnicas para prover melhorias no desenvolvimento de software com objetivos baseados em cumprimento de prazos e orçamentos assim como a implementação de produtos que atendem parâmetros de qualidades desejados. Essas melhorias são observáveis em diferentes pontos, desde o processo ao produto, cujos objetivos e prioridades podem variar de acordo com a metodologia de desenvolvimento.
%
Apesar de suas diferenças conceituais e de valores, a maior parte dos métodos preveem processos e técnicas referentes ao \emph{design}, testes e medição, que visam garantir a qualidade do software em desenvolvimento.


No contexto de projetos que adotam metodologias ágeis, observa-se que tanto a qualidade interna quanto a qualidade externa do software são preponderantes, pois são fatores fundamentais para suportar a simplicidade, o \emph{feedback} contínuo e adaptação à mudanças, valores que solidificam o desenvolvimento ágil.
%
A qualidade interna do software é observada a partir de atributos de qualidades na perspectiva de desenvolvimento que, segundo Berander (\citeyear{berander2005}), se resumem em corretude, testabilidade, flexibilidade, portabilidade, reusabilidade, interoperabilidade, analisabilidade, adaptatividade e estabilidade. As práticas ressaltadas pela metologia \emph{Extreme Programming} \cite{beck2000} visam realçar os valores dos atributos destacados, atributos esses que definem um bom \emph{design}.

%
O \emph{design} simples pode ser obtido através de técnicas como o Desenvolvimento Orientado à Testes \cite{beck2002} e a Refatoração \cite{fowler1999}, que por sua vez influenciam diretamente os atributos testabilidade, reusabilidade e adaptatividade. Ambas as técnicas se baseiam fortemente em testes unitários que provêem a segurança necessária para realização de mudanças assim como o feedback automatizado da manutenção do software.
%
A Programação em Pares, dentre outras práticas, também contribui para a garantia da qualidade interna, uma vez que exercita a programação e revisão ao mesmo tempo, reduzindo a ocorrência de não-conformidades técnicas e inserção de \emph{bugs}. Por outro lado, a qualidade externa do software pode ser alcançada a partir do envolvimento do ``cliente'' ao longo das atividades de desenvolvimento e, principalmente, a partir de entregas contínuas de software com valor de negócio.


Valores semelhantes podem ser observados nas comunidades de desenvolvimento de softwares livres refletindo diretamente na alta qualidade do código produzido em diversos projetos livres \cite{schmidt2001}; \cite{halloran2002}; \cite{michlmayr2003}.
%
Essas comunidades adotam a padronização de código e testes automatizados para manter a qualidade interna do código e incentivar a contribuição de diversos desenvolvedores.

A melhoria da qualidade interna do código apoia a melhoria contínua do processo oferecendo subsídios para que a equipe de desenvolvimento aumente sua produtividade e implemente novas funcionalidades com maior facilidade. Beck (\citeyear{beck2007}) corrobora essa afirmação ao destacar que a maior parte do tempo utilizado por um Programador ao inserir novas funcionalidades é destinado ao entendimento do código em manutenção.
%
Diretamente relacionado à qualidade de código está a sua segurança \cite{tsipenyuk2005}. A segurança de software está relacionada a confidencialidade, disponibilidade e integridade dos diversos componentes que compõe o software.

Dados do ICAT/NIST
\footnote{ ICAT foi um motor de busca de vulnerabilidades, desenvolvido pelo NIST(\emph{National Institue of Standards and Technology}), catalogadas no padrão CVE. O ICAT foi substituido pelo NVD (\emph{National Vulnerability Database }), que além de possuir o mesmo mecanismo de busca, é um repositório governamental dos Estados Unidos que armazena diversas informações sobre vulnerabilidades de software (nomenclaturas, métricas, checklists, etc).} de 2005 já apontavam que 80\% das vulnerabilidades remotamente exploráveis estavam ligadas a má codificação do programa \cite{duarte2005}.
%
Embora a segurança de uma aplicação também estejam relacionadas a aspectos externos ao software como a redes e componentes de hardware, o elo mais fraco continua sendo o próprio software. Dessa forma, cabe aos projetistas e desenvolvedores a responsabilidade do desenvolvimento de software seguro, sem prejuízos aos seus usuários.

%

À medida que o tempo vai passando, novas vulnerabilidades vão sendo descobertas pela comunidade. O projeto CVE (\emph{Common Vulnerabilities and Exposures List}), que tem como objetivo enumerar vulnerabilidades de software existentes, tinha uma lista de 321 vulnerabilidades diferentes no ano de sua concepção, em 1999 \cite{cve2002}.
%
No ano de 2002, a lista já havia aumentado para 2032 vulnerabilidades e atualmente o número já chega a 61 mil vulnerabilidades específicas encontradas por empresas de todo o mundo.


Visto esse cenário de inúmeras vulnerabilidades é fundamental que todos aqueles envolvidos no processo de produção do software tenham conhecimento das implicações relativas a segurança.
%
O conhecimento de vulnerabilidades e meios de detectá-las são habilidades necessárias para garantia de software seguro.
%
Esse conhecimento deve estar alinhado às habilidades de concepção de um bom \emph{design} para prover a segurança necessária no desenvolvimento do software e se alcançar os valores e objetivos anteriormente destacados.


Neste sentido, a medição pode ser utilizada como um processo de apoio ao acompanhamento da segurança e qualidade, através do estabelecimento de metas e indicadores que indiquem oportunidades de melhorias observáveis do produto.
%
Em um cenário otimista, os próprios Engenheiros de Software podem adotar como prática a medição do código-fonte para auxiliar as tomadas de decisões, ou até mesmo para avaliação do código inserido ou da aplicação de refatorações.
%
Entretanto, uma grande quantidade de métricas, coletas manuais e poucos recursos de visualização são fatores que acabam por desmotivar o uso dessas para o monitoramento do código.
%
Além disso, a compreensão do significado de valores obtidos através de métricas não é uma tarefa trivial, demandando um grande esforço de interpretação necessárias para a tomada de decisão efetiva sobre o projeto de software.


Assim, destaca-se a utilidade de ferramentas que auxiliem o processo de medição, compreensão e visualização do software.
%
Atualmente existem algumas ferramentas que automatizam a extração de métricas do código-fonte com objetivo de coletar as informações sobre o produto a partir da análise estática do código, as quais definimos como Extratores.
%
Outras ferramentas denominadas Plataformas de Monitoramento de código-fonte procuram oferecer melhores formas de monitoramento e visualização do software a partir da personalização de métricas e mecanismos que facilitem a interpretação dos resultados obtidos.
%
Alternativamente, um ambiente de \emph{Data Warehousing} (DWing) é uma solução que tem se destacado no ramo de \emph{Business Intelligence} - BI e tem ênfase em fornecer uma ambiente de fácil acesso a informação para a tomada de decisão. O \emph{Data Warehouse} constitui-se de uma base de dados que procura de maneira eficiente e flexível tratar de grande volume de dados e obter informações que auxiliem no processo de tomada de decisão \cite{lopes2007}. Alguns trabalhos já utilizaram um  DWing no contexto de monitoramento de métricas e apresentaram bons resultados  \cite{Castellanos2005} \cite{Folleco2007} \cite{Silveira2010}\cite{mazuco2011} \cite{rego2014}.

%

Portanto, neste trabalho foram exploradas a utilização de métricas para o monitoramento de código-fonte para compreender e estabelecer possíveis relações existentes entre as mesmas no que diz respeito a vulnerabilidades e qualidade de software. A partir do estabelecimento de cenários buscou-se identificar oportunidades de utilização de métricas na melhoria contínua do desenvolvimento e, consequentemente, na qualidade interna do produto. Com o objetivo de facilitar a interpretação e evitar possíveis equívocos, que são baseados em análises errôneas sobre métricas isoladas, correlações inexistentes ou até mesmo a escolha de métricas inadequadas cujo problemas são discutidos em \cite{chidamber1994}, estes cenários foram compostos a partir da análise de relação entre métricas. Tal relação buscou evidenciar boas e más características de bom \emph{design} de um projeto que impactam em vulnerabilidades no sistema.
%
Para auxiliar no monitoramento e na tomada de decisão, foi explorado o uso de plataforma de monitoramento de código-fonte Mezuro e um ambiente de \emph{DWing}.


%------------------------------------------------------------------------------%

\section{Problema}

%
Como utilizar métricas de código-fonte de maneira mais eficiente para melhor apoiar práticas de melhoria contínua da qualidade interna do software.

\section{Objetivos}

O objetivo deste trabalho consiste na proposta e implementação da técnica de construção de Cenários de Decisões para monitoramento do código-fonte que consiste na caracterização de determinado componente do sistema baseado na sua observação através de métricas, para orientar a aplicação de práticas e tomada de decisões sobre o software. Além disso, tem-se como objetivo do trabalho a proposta de alguns cenários através do uso desta técnica sobre características de segurança de software, a partir do estudo teórico sobre métricas de monitoramento de código-fonte. Por fim, o último objetivo consiste na realização de dois estudos de casos que visam reproduzir a técnica em duas ferramentas distintas de tomada de decisão, o Mezuro e um ambiente DWing.

Para alcançar os objetivos descritos, tem-se como objetivos específicos a realização das seguintes contribuições:

\section{Contribuições}
%
\begin{description}
	\item [Contribuições Tecnológicas]\
\end{description}
		\begin{enumerate}
			\item \textbf{CT1} - Evolução da plataforma livre Mezuro de monitoramento de código-fonte:
				\begin{enumerate}
					\item Evolução da arquitetura do Mezuro para melhorar sua modularização e flexibilidade.
					\item Evolução da configuração de métricas do Mezuro
					\item Evolução de mecanismos de visualização de software do Mezuro
				\end{enumerate}
			\item \textbf{CT2} - Criação de um ambiente de \emph{Data Warehousing} para monitoramento dos cenários de decisão no contexto de vulnerabilidade de software:
	        	\begin{enumerate}
	        		\item Criação de modelo dimensional que, diferente de um modelo relacional utilizado para modelagem de banco de dados, permitiu melhor performance para processamento analítico dos dados.
	        		\item Implementação de mecanismos de extração, transformação e carregamento da base de dados a partir dos resultados das ferramentas de análise estática.
	        		\item Geração Cubo de dados para definição das agregações para a manipulação dos dados do \emph{Data Warehouse}.
					\item Configuração dos mecanismos de visualização do cubo e geração de relatórios.
	        	\end{enumerate}
		\end{enumerate}
\begin{description}
	\item [Contribuições Científicas]\
\end{description}
	 	\begin{enumerate}
			\item \textbf{CC1} - Catalogar definições teóricas a respeito dos principais conceitos relacionados à vulnerabilidades de software.
			\item \textbf{CC2} - Estudo teórico sobre a relação de vulnerabilidades de software com o \emph{design}
			\item \textbf{CC3} - Definição de cenários a partir de estudos teóricos para melhorar a interpretação e tomada de decisão sobre métricas estáticas de código-fonte.
	 	\end{enumerate}

%------------------------------------------------------------------------------%

\section{Metodologia e Pesquisa}

Dado o objetivo principal desta monografia, foram realizados estudos teóricos através de revisão bibliográfica para compreensão e definição dos conceitos básicos sobre características de \emph{design}  e de vulnerabilidades de software, referindo-se às contribuições científicas CC1 e CC2. Este estudo é o insumo principal para a concepção da técnica Cenário de Decisão.

%

Por fim, a  partir da definição e discussão a respeito da técnica Cenário de Decisão, baseado principalmente na revisão bibliográfica sobre \emph{design}, segurança e métricas estáticas de software, serão apresentados dois estudos de casos da evolução de ferramentas de monitoramento de código-fonte para suportar Cenários de Decisões. Os estudos foram realizados sobre as duas soluções propostas, Mezuro e ambiente DWing.
%
Essa atividade está relacionado às contribuições tecnológicas deste trabalho de conclusão de curso, onde técnicas, métodos e padrões de Engenharia de Software puderam ser aplicados.


%------------------------------------------------------------------------------%

\section{Organização do Trabalho}

Esta monografia está dividida em mais outros 4 capítulos. A seguir, o leitor encontrará o Capítulo \ref{cap-metrics} onde são estruturados e discutidos os principais conceitos relacionados à \emph{design} e segurança, principalmente relacionado à vulnerabilidades de software. No Capítulo \ref{cap-metrics-esw} são introduzidos os principais conceitos de métricas de software, principalmente métricas estáticas de código-fonte, além de serem listadas e explicadas métricas de \emph{design} e métricas de vulnerabilidades. Posteriormente, no Capítulo \ref{cap-cenarios}  serão apresentados o conceito de Cenários de Decisões e algumas propostas que podem ser utilizadas no monitoramento de projetos. No Capítulo \ref{cap-case-study} iremos apresentar dois estudos de casos relacionados à evolução de plataformas de monitoramento de código-fonte para a adoção de Cenários de Decisões. Neste mesmo Capítulo, serão definidas e descritas características, evoluções e configurações dos dois ambientes de tomada de decisão citados na Introdução: a plataforma de monitoramento de código-fonte Mezuro e o \emph{Data Warehousing}. Por fim, no Capítulo \ref{cap-consideracoesFinais} serão realizadas as principais considerações e ponderações a respeito deste trabalho, além de apresentar possíveis passos futuros para a evolução da técnica Cenário de Decisão e das ferramentas envolvidas.

No fim desta monografia existem três apêndices que complementam e detalham os aspectos do presente trabalho. No Apêndice \ref{Att:design-security} são aprofundados em mais detalhes todos os conceitos estudos sobre \emph{design} e segurança de Software. O Apêndice \ref{Att:dw} apresenta detalhes técnicos sobre DWing, assim como Apêndice \ref{Att:mezuro} apresenta os detalhes técnicos sobre o Mezuro.
