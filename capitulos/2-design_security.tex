\chapter{Design e Segurança de Software}
\label{cap-metrics}

\section{Design de Software}
\label{sec-design-sw}

O desenvolvimento de software envolve diversas etapas que visam (i) o entendimento dos problemas a serem solucionados; (ii) o projeto de uma solução; (iii) a implementação desta solução; (iv) os testes sobre o produto, dentre outros passos.
%
As metodologias mais conhecidas de desenvolvimento de software tais como o Processo Unificado e o \emph{Extreme Programming} perpassam por essas etapas com diferentes focos e técnicas, ambos buscando a entrega de soluções em software que atendam aos problemas de seus clientes.
%
Com isso, tanto do ponto de vista processo quanto do produto de software, prazos e custos não devem ser os únicos fatores considerados para reger um projeto de software. Em complementação, a qualidade deve ser preponderante para o sucesso do produto que, além de resolver o problema, deve fazê-lo adequadamente segundo os atributos e critérios de qualidade estabelecidos para o projeto.

%

O \emph{design} ganha um grau de importância maior a medida que a complexidade dos softwares aumentam durante o desenvolvimento, pois suas consequências são diretas sobre os atributos de qualidade do software tais como flexibilidade, testabilidade, manutenibilidade, desempenho e segurança. Em projetos de software livre, por exemplo, esses atributos de qualidade são fatores fundamentais na atratividade de colaboradores para os projetos, onde observa-se uma correlação entre métricas de qualidade de código-fonte com a atratividade desses projetos \cite{meirelles2010sbes}. 


Visando amenizar os riscos de se construir um sistema que não alcance seus objetivos, a arquitetura do software tem recebida uma maior atenção através dos métodos, práticas de desenvolvimento e estudos acadêmicos.
%
O conjunto de decisões sobre as estruturas estáticas do sistema, hierarquia de módulos, descrição de dados, seleção de algoritmos, agrupamento e interface entre módulos podem ser previamente pensados a partir de modelos e documentação como também podem emergir a partir da aplicação de padrões de implementação e \emph{refactorings} sobre o código.
%
O mais importante é que a medida que novas funcionalidades são incorporadas, o \emph{design} do código continue mantendo suas características, aplicando bons princípios e não sendo um impendimento para manutenção e evolução do software, comprometendo assim sua existência e utilização.
%
Nesse sentido, o Engenheiro de Software tem a responsabilidade de desenvolver o software sem degradar sua arquitetura, respeitando padrões estabelecidos, evoluindo seu \emph{design} à medida que implementa novas linhas de código e manter o código limpo e seguro para possibilitar que outros Engenheiros também compreendam e evoluam o software.

%

Nesta Seção apresentaremos os princípios reconhecidos de ``bom \emph{design}'' e algumas formas como esses princípios podem ser aplicados no código-fonte. No Anexo \ref{Att:design-security}, identificamos como as decisões de design são observáveis em termos de \emph{Bad Smells} e Código Limpo.
%
Nesse contexto, pretendemos estudar e revisar como essas características observáveis possuem relação com métricas de código-fonte com objetivo de prover mecanismos que permitam ao Engenheiro de Software e gestores destinarem seus esforços na remoção de não-conformidades e evolução do software.

\subsection{O Design e seus Princípios}
\label{sec-principles-practises}

O \emph{design} do software consiste no conjunto de decisões importantes tomadas sobre a organização de um sistema de software que podem ser observadas e mapeadas em diferentes níveis de abstração no código-fonte e em outros produtos.
%
Katki (\citeyear{katki1991}) completa a definição explicando que \emph{design} é tanto o processo de definição da arquitetura, módulos, interfaces e outras características de um sistema quanto o resultado deste processo.
%
Para comunicação e documentação pode-se ter modelos que representem o \emph{design} conceitual demonstrando, por exemplo, o estilo arquitetural adotado na aplicação.
%
Por outro lado, esse modelo também é observável no código-fonte, assim como violação de restrições estabelecidas. Um nível mais detalhado do \emph{design} consiste nas decisões de implementações existentes no código de uma classe do software que influenciam, por exemplo, na testabilidade desta classe. Assim, as decisões de \emph{design} tratam problemas em diferentes níveis, desde da escolha do paradigma adequado para desenvolvimento do sistema até os padrões de nomenclatura a serem utilizadas no código.


Neste trabalho estamos essencialmente interessados nas decisões de projetos em nível de código e suas consequências no desenvolvimento do software.
%
Assim, estamos tratando principalmente do trabalho realizado pelo Engenheiro de Software, de quais formas esse trabalho é realizado e como podemos apoiar e evoluir a atuação desse profissional para obtenção de bons resultados para projetos de software.
%
Para o desenvolvimento de códigos com bom \emph{design}, assim como defendido por Spinellis (\citeyear{spinellis2006}), é importante que o Engenheiro de Software conheça os principais problemas de implementação conhecidos, características e princípios que compõem um paradigma e técnicas que permitem a aplicação desses princípios.

%

Neste trabalho estamos argumentando que os problemas do software podem ser definidos com as características indesejáveis que dificultam a manutenção e evolução do sistema ou até mesmo comprometem sua segurança.
%
Um dos mais conhecidos e facilmente observável é a complexidade do software que depende das estruturas de dados, tamanho do sistema, algoritmos utilizados, complexidade das estruturas de controle e do fluxo de dados do software \cite{basili1983}.
%
A complexidade afeta diretamente o quão compreensível um programa é, dificultando sua legibilidade e o encontro de \emph{bugs} e vulnerabilidades.
%
Além disso, a complexidade afeta a testabilidade dos módulos, dificultando uma boa cobertura de testes que exercitem as estruturas do software.
%
De modo geral, entendemos a complexidade como um fator de risco do software uma vez que a evolução do mesmo é comprometida por falta de legibilidade, flexibilidade e, muito provavelmente, pela falta de testes automatizados que suportem operações de \emph{refactorings} no código, aumentando os riscos do projeto em termos de qualidade, custos e prazos.


Outras características que argumentamos como indesejadas para o software surgem a partir da atribuição inadequada de responsabilidades entre os módulos que o compõe.
%
A baixa coesão surge quando um módulo é responsável pela realização de mais tarefas, concentrando mais dados e operações do que deveria. A baixa coesão gera problemas de falta de reusabilidade, aumenta a complexidade e dificulta a manutenção uma vez que não apoia a modularização adequadamente.
%
O alto acoplamento também pode ser consequência da baixa coesão e ocorre quando há forte dependência entre os componentes do software.
%
As consequências do alto acoplamento estão principalmente nas dificuldades de se manter e evoluir o software já que as modificações em um componente podem afetar outros.
%
Além disso, algumas mudanças tendem a deixar de serem simples, pois ocorrem em mais de um lugar, o que pode dificultar, por exemplo, a criação de testes unitários, uma vez que os componentes tendem a não poder serem testados isoladamente \cite{martensson2005}.


Os problemas elencados acima, assim como outras características indesejadas para o código-fonte são evitadas a partir da adoção de princípios reconhecidos e propostos através de diversos trabalhos para concepção de bons \emph{designs} \cite{martin2002} \cite{lakos1996} \cite{demeyer2002} \cite{lieberherr1996}. 
%
Esses princípios podem ser genéricos, como características desejáveis em qualquer código-fonte, ou podem estar relacionados com um paradigma específico, acoplado as estruturas introduzidas pelo próprio paradigma.
%
A seguir apresentamos os princípios gerais que indicamos ser levados em consideração pelos Engenheiros de Software no desenvolvimento de qualquer aplicação, pois são relacionados com a manutenibilidade, extensibilidade e testabilidade do software:

\begin{itemize}

\item \textbf{Reusabilidade} - O princípio de reusabilidade de código visa a implementação de componentes reutilizáveis, apoiados pela alta coesão e baixo acoplamento.
%
A reusabilidade pode existir em diferentes níveis do \emph{design}, podendo ser aplicada desde a introdução de funções reutilizáveis até serviços completos que oferecem operações autocontidas e reutilizáveis, sendo que a complexidade de implementação e as habilidades necessárias para conseguir a reusabilidade crescem proporcionalmente com o nível de abstração como discutido em \cite{cruise2007}.

Este princípio possui impactos positivos na qualidade do software, assim como no custo e produtividade do projeto uma vez que menos código deve ser produzido e mantido, menor o esforço de teste, dentre outros benefícios \cite{sametinger1997}.
%
O princípio de reusabilidade é estendido e aplicável na definição do conhecido princípio \emph{DRY - Don't Repeat Yourself}\footnote{\url{http://c2.com/cgi/wiki?DontRepeatYourself}} que é bastante enfatizado em \emph{frameworks} modernos de desenvolvimento web como Rails\footnote{\url{http://rubyonrails.org/}} e Django\footnote{\url{https://www.djangoproject.com/}}.
%
\item \textbf{Modularização} - O princípio de modularização visa a decomposição do sistema em estruturas lógicas bem definidas conceitualmente e fisicamente.
%
A modularização é muito importante para os atributos de qualidade do software, principalmente manutenibilidade e testabilidade, uma vez que apoia a alta coesão, baixo acoplamento e a reusabilidade.
%
Baldwin \& Clark (\citeyear{baldwin2000}) argumentam que um sistema bem modularizado permite o trabalho paralelo em diversas partes do produto, ameniza as dificuldades com a complexidade e esconde as incertezas e detalhes não necessários dentro dos módulos.
%
\item \textbf{Abstração} - O princípio de abstração recomenda que um elemento que compõe o \emph{design} seja representado apenas por suas características essenciais, provendo apenas as informações relevantes  para sua utilização, além de permitir sua distinção com outros elementos por parte do observador \cite{germoglio2009}.
%
A abstração é importante para a comunicação, compreensão e reutilização dos componentes.

\item \textbf{Baixo acoplamento} - O acoplamento é uma característica natural do software sendo o grau de qualquer interação existente entre dois ou mais módulos.
%
Essa característica é fundamental para a composição lógica do sistema, sendo que as interações dos componentes são necessárias para a implementação de funcionalidades que se complementam e juntas fornecem serviços complexos e completos.
%
Beck \& Diehl (\citeyear{diehl2011}) apresentam e discutem os diferentes tipos de acoplamentos e suas consequências na modularização do \emph{design}.
%
O princípio de baixo acoplamento ressalva a importância que esse acoplamento não seja forte ao ponto de dificultar a evolução de componentes que dependem de outros.
%
O grau de acoplamento determina como é difícil fazer alterações em uma aplicação, assim como quão difícil é compreendê-la e testá-la, principalmente se os componentes acoplados são instáveis e sofrem constantes mudanças.
%
Portanto, este princípio consiste na composição e modularização de serviços pouco acoplados com outros módulos a partir da melhor definição e distribuição de dados, de interfaces e responsabilidades.

\item \textbf{Alta coesão} - De acordo com a Terminologia Padrão da IEEE\footnote{Institute of Electric and Electronic Engineers} \cite{ieee1990}, coesão é o grau com que cada tarefa realizada por um módulo está relacionado funcionalmente com o mesmo.
%
Um módulo pode ser definido como coeso se todos as suas operações estão relacionadas com uma única abstração.
%
O princípio da alta coesão sugere manter o maior nível de coesão possível no \emph{design} de componentes do software.
%
A alta coesão apoia a redução de complexidade do sistema, pois melhora seu entendimento conceitual, diminui as dependências de seus módulos e apoia a modularidade e reusabilidade, além de possuir uma relação direta com o baixo acoplamento, conforme estudado por Baig (\citeyear{baig2004}).

\item \textbf{Simplicidade} - Um dos maiores desafios no desenvolvimento de software é manter o \emph{design} o mais simples possível, sendo esse o principal objetivo do princípio da simplicidade.
%
A simplicidade dentro do software é obtida a partir da redução da complexidade de seus módulos, desde a escolha de algoritmos até suas interfaces de comunicação.
%
Os benefícios deste princípio consiste na solução dos problemas inerentes a complexidade do software, discutidos anteriormente nesta Seção.
%
A simplicidade do software é também conhecida através do princípio \emph{KISS - Keep It Simple, Stupid} que enfatiza que a principal característica do \emph{design} deve ser a simplicidade.

\end{itemize}

No Apêndice \ref{Att:design-security} tem-se uma discussão mais aprofundada sobre as diferentes práticas e técnicas que podem ser utilizadas para a aplicação dos princípios de bom \emph{design} apresentados.

%========================================================================================%

\section{Segurança de Software}
\label{sec-metrics-security}
%introdução

A segurança de software está relacionada com o contínuo processo de manter a confidencialidade, integridade e disponibilidade nas diversas camadas que o compõe, sendo considerado parte dos requisitos não-funcionais do sistema. Independentemente da criticidade do sistema, a segurança em software deve ser tratada com prioridade dentro do ciclo de vida de desenvolvimento do software. Aggarwal e colaboradores (\citeyear{aggarwal2002}) citam que o custo e esforço gastos na segurança do software são bem altos, podendo chegar a 70\% do esforço total de desenvolvimento e suporte do software.

%

Problemas de segurança são recorrentes em diversos tipos de sistemas podendo gerar perdas materiais e humanas em diferentes proporções. Vulnerabilidades em softwares são as maiores causas de infecção de computadores das corporações e perda de dados importantes segundo a pesquisa Global Corporate IT Security Risks 2013 conduzido por B2B International em colaboração com Kaspersky Lab \cite{b2binternational2013}. Esse estudo aponta que aproximadamente 85\% das empresas reportaram incidentes internos de segurança de TI. Mesmo com o grande esforço destinado a aspectos de segurança, tais problemas são difíceis de solucionar, pois a Engenharia de Segurança de Sistemas está em fase intermediária de desenvolvimento \cite{pascoa2002}. Gandhi e colaboradores (\citeyear{gandhi2013}) realçam as dificuldades de se detectar vulnerabilidades no estágio operacional do software, pois os problemas de segurança não são endereçados ou suficientemente conhecidos nas fases iniciais do desenvolvimento de software. 

%

Formalmente, uma vulnerabilidade pode ser definida como uma instância de uma falha na especificação, desenvolvimento ou configuração do software de tal forma que a sua execução pode violar políticas de segurança, implícita ou explícita \cite{krsul1998}. Vulnerabilidades podem ser maliciosamente exploradas para permitir acesso não autorizado, modificações de privilégios e negação de serviço. A exploração maliciosa de vulnerabilidades em grade parte são realizadas através de \emph{Exploits}, ferramentas ou scripts desenvolvidos para este propósito, que se baseiam extensivamente nas vulnerabilidades mais comuns tal como \emph{buffer-overflow}. 

%

Vulnerabilidades podem existir em diferentes níveis de um sistema, podendo, portanto, gerar problemas com diferentes proporções. Os níveis mais comuns suscetíveis a existência de vulnerabilidades são:

%

\begin{itemize}
\item \textbf{Hardware} - Vulnerabilidades relacionadas ao hardware de sistemas que estão expostos a umidade, poeira, calor, locais inseguros, dentre outros fatores físicos relacionados ao local onde se encontra a infraestrutura de TI.

\item \textbf{Software} - Vulnerabilidades relacionadas às estruturas internas do software assim como aos dados que são acessados e processados. No geral, podem ser exercitados a partir de interações com o usuário não esperadas ou não validadas.

\item \textbf{Rede} - Vulnerabilidades relacionadas aos componentes da rede, tanto físicos (cabos, \emph{switches}) quanto em software (protocolos, dados). Este tipo de vulnerabilidade também está relacionada à falhas na comunicação como linhas de comunicação não protegidas, compartilhamento de informações com não interessados.

\item \textbf{Humana} - Vulnerabilidades relacionadas à processos que envolvem pessoas e níveis de acesso.

\item \textbf{Organizacional} - Vulnerabilidades relacionadas problemas em nível organizacional, principalmente relacionado à falta de políticas, auditorias e planos adequados.
\end{itemize}

%

No presente trabalho, estamos interessados essencialmente em vulnerabilidades de software. Mais especificamente, procuramos uma abordagem que facilite o tratamento destas vulnerabilidades dada sua importância e suas consequências. Nesse sentido, faz-se necessário compreender quais são as ocorrências conhecidas de falhas de segurança em software e com quais vulnerabilidades estas falhas se relacionam.

%

Vulnerabilidades de software são, na maior parte das vezes, causadas pela falta ou imprópria validação das entradas realizadas pelo usuário. Essas condições indesejáveis são usadas por usuários maliciosos para injetar falhas e códigos no sistema que os permitam executar seus próprios códigos e aplicações  \cite{jimenez2009}. McGraw e colaboradores (\citeyear{mcgraw2004}) afirmam que 50\% dos problemas de segurança surgem no nível de \emph{design}. Poucas ações específicas são tomadas por Engenheiros de Software para manter a segurança no desenvolvimento de novas funcionalidades ou até mesmo na realização de \emph{refactorings}. Em outras palavras, muitas vezes o desenvolvedor de software pode estar inserindo vulnerabilidades no código que podem ser exploradas por usuários maliciosos ou, acidentalmente, por usuários comuns. Mesmo os Engenheiros de Software que realizam testes unitários automatizados tendem a não exercitar estas vulnerabilidades, pois no geral testam principalmente as condições de uso padrão do software, enquanto deveriam explorar melhor o comportamento do software à interações indesejadas \cite{vries2006}.

%

Algumas das vulnerabilidades mais comuns estão listada abaixo:

%

\begin{itemize}
\item \textbf{\emph{Buffer overflow}}: caso comum de violação de segurança da memória que ocorre normalmente quando dados são escritos em \emph{buffers} de tamanhos fixos e ultrapassam os limites de memória definidos para eles. Como consequência, pode gerar mal funcionamento do sistema, já que o dado escrito pode corromper os dados de outros \emph{buffers} ou até mesmo de outros processos, erros de acesso à memória, resultados incorretos e até mesmo interromper a execução do software. Esta vulnerabilidade também pode ser explorada para injetar códigos maliciosos, alterando a ordem de execução do programa para que o código malicioso tome controle do sistema. Algumas linguagens de programação oferecem mecanismos de proteção contra acesso ou sobrescrita da dados em qualquer parte da memória indesejada. Contudo, \emph{buffer ovewflows} ocorrem principalmente com programas escritos em C e C++ que não realizam a verificação automática se o dado a ser escrito em um \emph{array} cabe dentro dos limites de memória do mesmo.
%Podemos criar listagem de código 

\item \textbf{\emph{Dangling pointer}}: vulnerabilidade de violação de segurança da memória que ocorre quando um ponteiro não aponta para um objeto ou destino válido. Esta vulnerabilidade acontece ao se deletar um objeto ou desalocar a memória de um ponteiro sem modificar, entretanto, o valor deste ponteiro. Como resultado, o ponteiro ainda aponta para a mesma posição de memória que, por sua vez, já não está mais alocada para esse processo. Como consequência, o sistema operacional pode realocar essa posição de memória para outro processo que, se acessado novamente pelo primeiro processo, irá conter dados inconsistentes com o esperado. Em C e C++ esta vulnerabilidade existe também quando o ponteiro de um endereço de memória é declarado somente no escopo de um função e retornado por esta função. Muito provavelmente esse endereço de memória será sobrescrito na pilha de alocação do processo pela chamada de funções posteriores. Além de inconsistência de dados, esta vulnerabilidade pode ainda ser a causa de quebras de programas, como falhas de segmentação e pode ser explorada por ataques de injeção de código \cite{afek2007}. Algumas linguagens de programação como Java, Python e Ruby possuem um mecanismo de gerenciamento de destruição de objetos chamado \emph{Garbage Collector}\footnote{\url{http://www.informit.com/articles/article.aspx?p=30309&seqNum=6}}. 
%informações importantes sobre esta vulnerabilidade: https://www.usenix.org/legacy/events/sec10/tech/full_papers/Akritidis.pdf
%Seria interessante manter este footnote ou melhor referenciar como um Pattern do livro: http://www.informit.com/store/real-time-design-patterns-robust-scalable-architecture-9780201699562?w_ptgrevartcl=Real-Time+Design+Patterns%3a+Memory+Patterns_30309
%Listagem do código 

\item \textbf{\emph{Strings} formatadas não-controladas}: vulnerabilidade decorrida do tratamento inadequado das entradas do usuário sobre o software que, quando explorada, o dado submetido por uma \emph{string} de entrada é avaliado como um comando pela aplicação. Uma \emph{string} formatada pode conter dois tipos de dados: caracteres imprimíveis e diretivas de formatação de caracteres. Na linguagem C, funções de \emph{strings} formatadas tal como o \emph{printf} recebem um número variável de argumentos, dos quais uma \emph{string}  formatada é obrigatória. Para acessar o restante dos parâmetros que a chamada da função colocou na pilha, a função de \emph{string} formatada analiza a sequência de formatação e interpreta as diretrizes do formato a medida que realiza sua leitura \cite{lhee2002}.

\item \textbf{\emph{SQL Injection}}: vulnerabilidade presente em aplicações que aceitam dados de uma fonte não confiável, não os validando adequadamente e os usando posteriormente para construção de \emph{queries} dinâmicas de SQL para comunicação com o banco de dados da aplicação. Todos os tipos de sistemas que incorporam SQL estão sujeitos a esta vulnerabilidade, apesar de serem mais comuns em aplicações WEB. Como consequência da exploração desta vulnerabilidade tem-se a perda de confiabilidade e quebra de integridade dos dados de uma base de dados. Em alguns casos, a exploração de \emph{SQL Injection} pode permitir ao atacante levar vantagens através da persistência de informações e geração de conteúdos dinâmicos em páginas web \cite{uscert2012}.
\end{itemize}

%Lista: https://cwe.mitre.org/top25/
%http://nvd.nist.gov/cwe.cfm

\subsection{Classificações e taxonomias de vulnerabilidades}
\label{subsec-vulnerabilities-taxonomy}
%

O primeiro passo para que o desenvolvedor consiga cuidar de vulnerabilidades no código fonte é conhecer quais são os problemas mais comuns existentes em softwares e como os atacantes utilizam estas vulnerabilidades para falhar o sistema. Existem muitos tipos de vulnerabilidades, e com o passar do tempo, novas vulnerabilidades são descobertas e novos \emph{exploits} são criados por atacantes. Em meio a essa diversidade, a classificação de vulnerabilidades representa um grande desafio, porém, já houve vários avanços na área com o objetivo de enumerar e catalogar estas vulnerabilidades.

%

O CVE (\emph{Common Vunerabilities and Exposures}) é um projeto criado pelo MITRE\footnote{\url{http://www.mitre.org/}} com o objetivo de enumerar as vulnerabilidades descobertas pela comunidade para facilitar o compartilhamento de informações. Como é mostrado em \cite{cve2001martin}, antigamente cada organização nomeava de sua maneira uma vulnerabilidade, de forma que uma mesma vulnerabilidade era referenciada de maneira completamente distinta entra as organizações. 

%

A enumeração realizada pelo CVE é feita por uma junta de especialistas 	de segurança. Essa equipe nomeia, descreve e referencia cada nova ameaça. As novas ameaças são reportadas pela comunidade. Após estudo e aprovação, essa ameaça passa a incluir a lista da CVE, cujo o identificador é representado da seguinte forma: CVE-2014-0002. O valor 2014 significa que a CVE foi criada em 2014, e o numero 0002 se refere ao numero sequencial atribuído a essa CVE dentre todas que foram aprovadas nesse ano.

%

Surgiram propostas além do CVE para a classificação e agrupamento de vulnerabilidades, que são chamadas de taxonomias. O trabalho de Malerba (\citeyear{malerba2010}) descreve as as primeiras propostas taxonômicas, como os trabalhos de Landwher em 1992(\emph{A toxonomy of Computer Security Flaws}) e Aslam em 1997 (\emph{Use of  a taxonomy of Security Faults}). Porém, consideramos mais relevante abordar mais detalhadamente sobre as taxonomias mais recentes, que são utilizadas no projeto CWE, iniciativa semelhante a CVE que será abordada mais adiante. As principais taxonomias são:

%

\begin{itemize}
\item PLOVER (\emph{Preliminary List Of Vulnerability Examples for Researchers})
\item CLASP (\emph{Comprehensive, Lightweight Application Security Process})
\item \emph{Seven Pernicious Kingdom}
\end{itemize}

%

O PLOVER foi criado em 2005 pelo MITRE em parceria com o DHS\footnote{US. Departament of Homeland Secutity} e o NIST\footnote{National Institute of Technology}. É um documento que lista mais de 1400 exemplos reais de vulnerabilidades identificados pelo CVE. Trata-se de um framework conceitual que descreve as vulnerabilidades em diversos níveis de detalhe. O PLOVER também fornece uma série de conceitos que podem ajudar na comunicação e discussões à respeitos de vulnerabilidades.

%

Foram definidas 28 categorias de mais alto nível para a categorização de vulnerabilidades. Algumas dessas são:

\begin{itemize}
\item BUFF: inclui vulnerabilidades de Buffer Overflow, formatação de strings, etc;
\item CRIPTO: inclui vulnerabilidades relacionadas a criptografia;
\item SPECTS: inclui vulnerabilidades que ocorrem e tecnologias específicas, como a injeção de SQL e XSS
\end{itemize}

A lista completa das categorias e mais detalhes sobre o PLOVER podem ser encontrados em \cite{christey2006}.

%

O CLASP não é apenas uma taxonomia de categorização de vulnerabilidades, mas sim um processo que busca melhorar a segurança de softwares. No de diz respeito a classificação, o CLASP define 6 categorias alto nível que incluem 104 tipos de vulnerabilidades, que são:

\begin{itemize}
\item Erros de tipo e de Range
\item Problemas no ambiente 
\item Erros de sincronização e de temporização
\item Erros de protocolo
\item Erros de lógica
\item \emph{Malware}
\end{itemize}

%
Exemplificando, as vulnerabilidades do tipo \emph{Buffer Overflow} se encaixariam na categoria de erro de tipo e de range, pois nesse tipo de vulnerabilidade é permitido a escrita de uma informação além o limite do buffer. A vulnerabilidade do tipo  injeção de SQL também se encaixaria nessa categoria, pois esse tipo de vulnerabilidade ocorre quando não há validação no tipo de informação fornecida pelo usuário. Mais detalhes podem ser vistos no site do CWE\footnote{\url{http://cwe.mitre.org/about/sources.html}}.

%

A taxonomia Seven Pernicious Kingdoms é a que possibilita melhor entendimento ao desenvolvedor, e é até utilizada como base em uma das visões das CWEs denominada \emph{Development View}\footnote{\url{http://cwe.mitre.org/data/graphs/699.html}}. Utiliza os conceitos da biologia de Reino e Filo. Nessa taxonomia, o Reino é a classificação mais abrangente e o Filo é uma subdivisão do Reino. Embora o nome sugere sete, possui oito reinos sendo que sete reinos são dedicados a vulnerabilidades de código fonte e um reino referente a aspectos de configuração e ambiente. São eles:

\begin{itemize}
\item \textbf{Validação e representação de entrada}: inclui erros de \emph{Buffer Overflow}, injeção de SQL, XSS, etc;
\item \textbf{Abuso de API}: Ocorre quando uma função que chama outra função assume certas condições que não estão garantidas pela rotina chamada.
\item \textbf{\emph{Features} de segurança}: está relacionado ao uso correto de peças chaves em segurança de código como criptografia, autenticação, gerenciamento de privilégios, etc;
\item \textbf{Tempo e estado}: está relacionado a erros de paralelismo, sincronização e uso de informação.
\item \textbf{Erros}: está relacionado erros oriundo da falta de tratamento de erros da aplicação.
\item \textbf{Qualidade de código}: são erros originados pela falta de qualidade no código fonte. Geralmente acontecem quando são utilizadas más praticas de programação que podem gerar colapso no sistema, como por exemplo, não desalocar recursos não utilizados pode gerar \emph{Memory Leak}\footnote{Falta de memória livre para uso no sistema};
\item \textbf{Encapsulamento}: são erros relacionados ao não estabelecimento de limite de acesso aos componentes do sistema
\item \textbf{Ambiente}: são erros que estão relacionados a fatores externos do software, que não estão no escopo deste trabalho.

\end{itemize}

A ideia dessa taxonomia foi criar reinos bem amplos para que novos filos fossem inseridos em seu lugar correto, porém a taxonomia proposta está aberta para a inserção de novos reinos, caso necessário. Mais informações e detalhes a cerca dos filos que incluem cada reino pode ser encontrado em \cite{tsipenyuk2005}

%
Com esse cenário apresentado acima, em que temos a CVE como projeto de enumerar as vulnerabilidades encontradas e as taxonomias elaboradas por diversos trabalhos com o objetivo de classificar e dar mais informações sobre a vulnerabilidade, ainda haviam a necessidade das empresas e organizações de utilizarem o uma terminologia padrão para listar e classificar vulnerabilidades de software, gerando uma linguagem unificada e uma base para ferramentas e serviços de medição dessas vulnerabilidades. Para essa necessidade foi elaborado o projeto CWE.

%

O CWE é uma lista formal de vulnerabilidades comuns de software (\emph{Common Weakness Enumeration}). Tem como objetivo estabelecer uma linguagem comum para descrever vulnerabilidades de software no \emph{design}, arquitetura ou no código; servir de base para ferramentas de análise de cobertura de segurança de código, dessa forma é possível saber quais vulnerabilidades as ferramentas conseguem capturar; e prover uma base de informações padrão a respeito de como identificar, mitigar e prevenir uma certa vulnerabilidade. Dessa forma, diferente da enumeração fornecida pela CVE, O CWE também inclui detalhes sobre a vulnerabilidade e cria uma classificação baseadas nos trabalhos desenvolvidos sobre taxonomias apresentados neste trabalho e outros que não foram apresentados. Além disso, observou-se que o CVE busca enumerar casos de vulnerabilidades reais na comunidade, especificando a maneira de como a vulnerabilidade foi explorada. Já o projeto CWE, o termo "vulnerabilidade" está mais relacionado a fraqueza de software, que se refere a códigos e práticas de programação que podem oferecer risco a segurança da aplicação, não indicando como a vulnerabilidade pode ser explorada e sim indicando quais características podem tornar o software vulnerável.

%

\subsection{Princípios de segurança}
\label{sec-security-principles}

Como visto no estudo sobre a classificação e taxonomias, existem muitas vulnerabilidades de software que são passíveis de exploração por atacantes. Nesse sentido, ganha-se importância para o desenvolvimento de softwares mais seguros que os Engenheiros de Software construam códigos com qualidade suficiente que os permitam identificar, corrigir e evitar a inserção de vulnerabilidades. Tendo-se o conhecimento de quais são as principais vulnerabilidades existentes, os Engenheiros podem tratar essas vulnerabilidades desde as primeiras fases de \emph{design} e desenvolvimento código-fonte, seguindo até o fim do ciclo de vida do desenvolvimento do software. Assim como os desenvolvedores programam aplicando ao código princípios de \emph{design}, devem evoluir o código aplicando princípios de \emph{design} seguro tais quais os apresentados por outros trabalhos \cite{saltzer1975} \cite{bishop2003} \cite{mcgraw2002} \cite{a1lshammari2009}.

%

Dentre os princípios de segurança que os Engenheiros de Software podem aplicar no \emph{design} de seus programas, introduziremos aqui os seguintes princípios:

\begin{itemize}
\item \textbf{\emph{Least privilege}}: este princípio sugere que o usuário deve ter somente os direitos necessários para completar suas tarefas \cite{bishop2003}. Em termos de \emph{design} de classes, significa que o \emph{design} mais seguro é aquele cujos métodos realizam o menor número de ações possíveis \cite{a1lshammari2009}.

\item \textbf{\emph{Reduce attack surface}}: este princípio tem como objetivo limitar o acesso a dados não permitidos. Pode-se aplicar este princípio reduzindo-se a quantidade de código executável, com \emph{design} prezando por menor números métodos de acesso (públicos) e menor número de parâmetros possíveis que possam afetar atributos privados para realização de uma tarefa. Pode-se também buscar eliminar serviços que são usados somente por poucos clientes.

\item \textbf{\emph{Defend in depth}}: este princípio sugere que os mecanismos de defesas devem ser aplicados na maior extensibilidade possível, mesmo que isso gere redundância. O princípio \emph{defend in depth} busca defender o sistema contra qualquer possível ataque através de implementação de métodos ou mecanismos diferentes de tratamento destes ataques. O \emph{design} em camadas facilita sua implementação, pois permite dividir os métodos de defesa de acordo com as responsabilidades de cada camada. Como ponto negativo, a implementação de vários mecanismos de defesa pode acrescentar complexidade ao software, aumentando riscos de inserção de outras vulnerabilidades e a dificuldade de encontrá-las.

\item \textbf{\emph{Fail securely}}: este princípio está relacionado ao controle das falhas que possam ocorrer na aplicação. As possíveis falhas existentes em um software devem ser exploradas e tratadas para que o software esteja preparado para responder a estas falhas adequadamente, sem gerar alarmes, quebrar a aplicação e principalmente abrir espaços para mais ataques maliciosos. Com a aplicação do princípio \emph{fail securely} tem-se a identificação e tratamento de erros, a inserção de mecanismos de respostas que facilitam a utilização correta do software e que permita que estes comportamentos sejam testados pelos desenvolvedores. Outra consequência positiva da aplicação deste princípio é que o princípio \emph{defend in depth} é apoiado uma vez que a identificação de possíveis erros reforça a modularização e separação dos métodos de tratamento dos mesmos.

\item \textbf{\emph{Economy of mechanism}}: princípio que se refere a manter o código que implementa mecanismos seguros menor e o mais simples possível. Este princípio é de suma importância para o tratamento de vulnerabilidades no desenvolvimento do software, pois a simplicidade é fundamental para que os Engenheiros de Software possam encontrar erros e corrigi-los. A medida que a complexidade aumenta, os módulos inseguros do software tendem a ficarem ocultos e mais difíceis de serem testados. A aplicação de técnicas de programação do XP tais como \emph{refactoring}, \emph{test-driven development} e programação em pares são fundamentais para alcançar os objetivos deste princípio. Este princípio está extremamente ligado ao princípio de design \emph{KISS - Keep It Simple, Stupid!}, pois ambos enfatizam que evitar a complexidade significa evitar problemas \cite{mcgraw2002}

\item \textbf{\emph{Mediate completely}}: princípio que defende que todos os acessos a quaisquer objetos devem ser verificados para garantir se há permissões para realizar tal ação. Se em algum momento for solicitado a leitura de um objeto, o sistema deve verificar se o sujeito tem permissão de leitura. Caso tenha, deve prover somente os recursos necessários para realização das tarefas que interessa a este sujeito. Essa operação deve se repetir todas as vezes que a requisição ao objeto for feita, não somente na primeira vez \cite{bishop2003}.

\item \textbf{\emph{Separation of duties}}: princípio relacionado a separação de interesses dentro dos métodos e mecanismos de segurança do software. A OWASP\footnote{Open Web Application Security Project} sugere a separação entre as entidades que aprovam a ação, entidades que realizam a ação e entidades que monitoram a ação \footnote{\url{https://www.owasp.org/index.php/Separation_of_duties}}. Este princípio está diretamente relacionado com os princípios de \emph{design} orientado à objetos tais como coesão e separação de interesses. Por outro lado, também mantém forte relação com o princípio de \emph{design} seguro \emph{Economy of mechanism}, pois proporciona código mais limpo e que podem ser mantidos separadamente.
\end{itemize}

Um estudo mais aprofundado sobre a relação existente entre os princípios de segurança e de \emph{design} pode ser encontrado no Apêndice \ref{Att:design-security}, na Seção \ref{sec-security-application}.

