\chapter{Design de Software e Métricas}
\label{cap-metrics}

\section{Métricas em Engenharia de Software}
\label{sec-metrics-esw} 
Tanto a existência de um bom design quanto a de testes automatizados que exercitem as funcionalidades são características desejáveis em projetos de software.	Boa parte das técnicas modernas da Engenharia de Software (CITAR TÉCNICAS) são voltadas para desenvolvimento com design que proporcione simplicidade, manutenibilidade e testabilidade. Mesmo que a maior parte dessas técnicas tenham sido disseminadas a partir do advento dos Métodos Ágeis e do Software Livre, cujo foco central está em atividades relacionadas ao código-fonte, elas são aplicáveis independentemente da metodologia de desenvolvimento utilizada \cite{meirelles2013metrics}. A valorização por softwares que atendam estes parâmetros de qualidade deve-se ao fato de sempre que o Engenheiro de Software está escrevendo novas linhas de código, um tempo significativo é gasto por ele na leitura e entendimento do código existente, muitas vezes desenvolvidos por outros Engenheiros. Martin (\citeyear{martin2008}) destaca que o código-fonte deve ser escrito para ser entendido principalmente por pessoas, e não pela máquina.

%

Neste sentido, o monitoramento da qualidade de código-fonte é fundamental e pode apoiar a utilização de técnicas de desenvolvimento que visam a melhoria contínua do código. Além disso, as métricas de código-fonte são muito importantes para projetos de software, pois estas podem ser utilizadas tanto como ferramenta para gestão do projeto quanto como referência técnica para tomada de decisões sobre o código-fonte.

%

Uma métrica, no âmbito da Engenharia de Software, provê uma forma de medir quantitativamente atributos relacionados as entidades do software e do processo de desenvolvimento. Assim, métricas são importantes ferramentas para avaliação da qualidade do código-fonte produzido e acompanhamento do projeto. Meirelles (\citeyear{meirelles2013metrics}) destaca que, com métricas de software, propõe-se uma melhoria de processo de gestão com identificação, medição e controle dos parâmetros essenciais do software.

%

Métricas de monitoramento de código-fonte possuem natureza objetiva e foram inicialmente concebidas para medir o tamanho e a complexidade do software \cite{henry1984kafura}\cite{troy1981zweben}\cite{yau1985zweben}. Outras métricas surgiram para avaliar softwares que utilizam paradigmas específicos, não sendo aplicáveis a qualquer tipo de software. Por exemplo, métricas orientada a objetos são usadas para avaliar sistemas orientados a objetos \cite{systa2000}. Métricas OO são destinadas, portanto, para avaliar a coesão de classes, as hierarquias de classes existentes, nível de acoplamento entre classes, reuso de código, dentre outras características.

%

Algumas características importantes ajudam a classificar as métricas de código-fonte. Assim, podemos classificá-las como estáticas e dinâmicas. Como o próprio nome diz, métricas estáticas capturam propriedades estáticas dos componentes de software e não necessita que o software seja executado para que seus valores sejam coletados. Por outro lado, métricas dinâmicas refletem características chaves tais como dependência dinâmica entre os componentes em tempo de execução do software.

%

As métricas de software também podem ser classificadas quanto ao método de obtenção. Métricas primitivas podem ser diretamente coletadas refletindo um valor observável de um atributo, sendo raramente interpretadas independentemente. Por outro lado, métricas compostas são obtidas a partir da relação de uma ou mais métricas, derivada, por exemplo, a partir de uma expressão matemática.

%

Entretanto, as definições de métricas adequadas para o acompanhamento do projeto, dimensionamento do software e principalmente para a aferimento da qualidade do código-fonte são tarefas que aumentam a complexidade de adoção de métricas em projetos de software, assim como destacado por Rakić e Budimac (\citeyear{rakic2011budimac}). Isto se deve a diversos fatores: à grande quantidade de métrica existentes; pouca aderência de algumas métricas com a realidade; diversas formas de interpretação de dados; dificuldades de definir parâmetros para comparação; poucos recursos de visualização de dados; coleta de dados não automatizados ou difíceis. Fenton e Pfleeger (\citeyear{fenton1998}) definem características desejáveis de métricas que orientam a escolha das mesmas enquanto outros autores \cite{meirelles2013metrics}\cite{almeida2010} estudam formas de viabilizar a utilização de métricas pelos desenvolvedores em geral. O presente trabalho visa correlacionar métricas de desenvolvimento de software com objetivo de definir configurações para estabelecer cenários que representem o estado da qualidade do software. Desta forma, espera-se reduzir as dificuldades de utilização de métricas de código fonte, tanto para o acompanhamento gerencial quanto para a tomada de decisões de design por desenvolvedores baseada em evidências. Para tanto, nas próximas seções serão apresentados estudos realizados sobre métricas de monitoramento de código-fonte para sistemas orientados à objetos, métricas para avaliação de vulnerabilidades do software e o estado da arte existente sobre a detecção de Bad Smells e Código Limpo a partir de métricas.

%

\section{Métricas e Design de Software}
\label{sec-design-sw}

Escrever Sub-seção

\section{Bad Smells}
\label{sec-bad-smells}

Escrever Sub-seção

\section{Código Limpo}
\label{sec-clean-code}

Escrever Sub-seção


\section{Métricas e Segurança de Software}
\label{sec-metrics-security}

A segurança de software está relacionada com o contínuo processo de manter a confiabilidade, integridade e disponibilidade nas diversas camadas que o compõe, sendo considerado parte dos requisitos não-funcionais do sistema. Independentemente da criticidade do sistema, a segurança em software deve ser tratada com prioridade dentro do ciclo de vida de desenvolvimento do software. Aggarwal e colaboradores (\citeyear{aggarwal2002}) cita que o custo e esforço gastos na segurança do software são bem altos, podendo chegar a 70\% to esforço total de desenvolvimento e suporte do software.

%

Problemas de segurança são recorrentes em diversos tipos de sistemas podendo gerar perdas materiais e humanas em diferentes proporções. Vulnerabilidades em softwares são as maiores causas de infecção de computadores das coorporações e perda de dados importantes segundo a pesquisa Global Corporate IT Security Risks 2013 conduzido por B2B International em colaboração com Kaspersky Lab \cite{b2binternational2013}. Este estudo aponta que aproximadamente 85\% das empresas reportaram incidentes internos de segurança de TI. Mesmo com o grande esforço destinado a aspectos de segurança, tais problemas são difíceis de solucionar, pois a Engenharia de Segurança de Sistemas está em fase intermediária de desenvolvimento \cite{pascoa2002}. Gandhi e colaboradores (\citeyear{gandhi2013}) realçam as dificuldades de se detectar vulnerabilidades no estágio operacional do software, pois os problemas de segurança não são endereçados ou suficientemente conhecidos nas fases iniciais do desenvolvimento de software. 

%

Formalmente, uma vulnerabilidade pode ser definida como uma instância de uma falha na especificação, desenvolvimento ou configuração do software de tal forma que a sua execução pode violar políticas de segurança, implícita ou explícita \cite{krsul1998}. Vulnerabilidades podem ser maliciosamente exploradas para permitir acesso não autorizado, modificações de privilégios e negação de serviço. A exploração maliciosa de vulnerabilidades em grade parte são realizadas através de \emph{Exploits}, ferramentas ou scripts desenvolvidos para este propósito, que se baseiam extensivamente nas vulnerabilidades mais comuns tal como \emph{buffer-overflow}. 

%

Vulnerabilidades podem existir em diferentes níveis de um sistema, podendo, portanto, gerar problemas com diferentes proporções. Os níveis mais comuns suscetíveis a existência de vulnerabilidades são:

%

\begin{itemize}
\item \textbf{Hardware} - Vulnerabilidades relacionadas ao hardware de sistemas que estão expostos a humidade, poeira, calor, locais inseguros, dentre outros fatores físicos relacionados ao local onde se encontra a infra-estrutura de TI.
\item \textbf{Software} - Vulnerabilidades relacionadas às estruturas internas do software assim como aos dados que são acessados e processados. No geral, podem ser exercitados a partir de interações com o usuário não esperadas ou não validadas.
\item \textbf{Rede} - Vulnerabilidades relacionadas aos componentes da rede, tanto físicos (cabos, \emph{switches}) quanto em software (protocolos, dados). Este tipo de vulnerabilidade também está relacionada à falhas na comunicação como linhas de comunicação não protegidas, compartilhamento de informações com não interessados.
\item \textbf{Humana} - Vulnerabilidades relacionadas à processos que envolvem pessoas e níveis de acesso.
\item \textbf{Organizacional} - Vulnerabilidades relacionadas problemas em nível organizacional, principalmente relacionado à falta de políticas, auditorias e planos adequados.
\end{itemize}

%

No presente trabalho, estamos interessados essencialmente em vulnerabilidades de software. Mais especificamente, procuramos uma abordagem que facilite o tratamento destas vulnerabilidades dada sua importância e suas consequências. Neste sentido, faz-se necessário compreender quais são as ocorrências conhecidas de falhas de segurança em software e com quais vulnerabilidades estas falhas se relacionam.

%

Vulnerabilidades de software são, na maior parte das vezes, causadas pela falta ou imprópria validação das entradas realizadas pelo usuário. Essas condições indesejáveis são usadas por usuários maliciosos para injetar falhas e códigos no sistema que os permitam executar seus próprios códigos e aplicações  \cite{jimenez2009}. McGraw e colaboradores (\citeyear{mcgraw2004}) afirmam que 50\% dos problemas de segurança surgem no nível de design. Poucas ações específicas são tomadas por Engenheiros de Software para manter a segurança no desenvolvimento de novas funcionalidades ou até mesmo na realização de \emph{refactorings}. Em outras palavras, muitas vezes o Desenvolvedor de software pode estar inserindo vulnerabilidades no código que podem ser exploradas por usuários maliciosos ou, acidentalmente, por usuários comuns. Mesmo os Engenheiros de Software que realizam testes unitários automatizados tendem a não exercitar estas vulnerabilidades, pois no geral testam principalmente as condições de uso padrão do software, enquanto deveriam explorar melhor o comportamento do software à interações indesejadas \cite{vries2006}.

%

O primeiro passo para que o Desenvolvedor consiga cuidar de vulnerabilidades no código-fonte é conhecer quais são os problemas mais comuns existentes em softwares e como os atacantes utilizam estas vulnerabilidades para falhar o sistema. Algumas das vulnerabilidades mais conhecidas e frequentes são:

%

\begin{itemize}
\item \textbf{\emph{Buffer overflow}}: caso comum de violação de segurança da memória que ocorre normalmente quando dados são escritos em buffers de tamanhos fixos e ultrapassam os limites de memória definidos para eles. Como consequência, pode gerar mal funcionamento do sistema, já que o dado escrito pode corromper os dados de outros buffers ou até mesmo de outros processos, erros de acesso à memória, resultados incorretos e até mesmo interromper a execução do software. Esta vulnerabilidade também pode ser explorada para injetar códigos maliciosos, alterando a ordem de execução do programa para que o código malicioso tome controle do sistema. Algumas linguagens de programação oferecem mecanismos de proteção contra acesso ou sobrescrita da dados em qualquer parte da memória indesejada. Contudo, \emph{buffer ovewflows} ocorrem principalmente com programas escritos em C e C++ que não realizam a verificação automática se o dado a ser escrito em um \emph{array} cabe dentro dos limites de memória do mesmo.
%Podemos criar listagem de código 
\item \textbf{\emph{Dangling pointer}}: vulnerabilidade de violação de segurança da memória que ocorre quando um ponteiro não aponta para um objeto ou destino válido. Esta vulnerabilidade acontece ao se deletar um objeto ou desalocar a memória de um ponteiro sem modificar, entretanto, o valor deste ponteiro. Como resultado, o ponteiro ainda aponta para a mesma posição de memória que, por sua vez, já não está mais alocada para este processo. Como consequência, o sistema operacional pode realocar esta posição de memória para outro processo que, se acessado novamente pelo primeiro processo, irá conter dados incosistentes com o esperado. Em C e C++ esta vulnerabilidade existe também quando o ponteiro de um endereço de memória é declarado somente no escopo de um função e retornado por esta função. Muito provavelmente este endereço de memória será sobrescrito na pilha de alocação do processo pela chamada de funções posteriores. Além de incosistência de dados, esta vulnerabilidade pode ainda ser a causa de quebras de programas, como falhas de segmentação e pode ser explorada por ataques de injeção de código \cite{afek2007}. Algumas linguagens de programação como Java, Python e Ruby possuem um mecanismo de gerenciamento de destruição de objetos chamado \emph{Garbage Collector}\footnote{\url{http://www.informit.com/articles/article.aspx?p=30309&seqNum=6}}. 
%informações importantes sobre esta vulnerabilidade: https://www.usenix.org/legacy/events/sec10/tech/full_papers/Akritidis.pdf
%Seria interessante manter este footnote ou melhor referenciar como um Pattern do livro: http://www.informit.com/store/real-time-design-patterns-robust-scalable-architecture-9780201699562?w_ptgrevartcl=Real-Time+Design+Patterns%3a+Memory+Patterns_30309
%Listagem do código 
\item \textbf{\emph{Strings} formatadas não-controladas}: vulnerabilidade decorrida do tratamento inadequado das entradas do usuário sobre o software que, quando explorada, o dado submetido por uma \emph{string} de entrada é avaliado como um comando pela aplicação. Uma \emph{string} formatada pode conter dois tipos de dados: caracteres imprimíveis e diretivas de formatação de caracteres. Na linguagem C, funções de \emph{strings} formatadas tal como o \emph{printf} recebem um número variável de argumentos, dos quais uma \emph{string}  formatada é obrigatória. Para acessar o restante dos parâmetros que a chamada da função colocou na pilha, a função de \emph{string} formatada analiza a sequência de formatação e interpreta as diretrizes do formato a medida que realiza sua leitura \cite{lhee2002}.
\item \textbf{\emph{SQL Injection}}: vulnerabilidade presente em aplicações que aceitam dados de uma fonte não confiável, não os validando adequadamente e os usando posteriormente para construção de \emph{queries} dinâmicas de SQL para comunicação com o banco de dados da aplicação. Todos os tipos de sistemas que incorporam SQL estam sujeitos a esta vulnerabilidade, apesar de serem mais comuns em aplicações WEB. Como consequência da exploração desta vulnerabilidade tem-se a perda de confiabilidade e quebra de integridade dos dados de uma base de dados. Em alguns casos, a eploração de \emph{SQL Injection} pode permitir ao atacante levar vantagens através da persistência de informações e geração de conteúdos dinâmicos em páginas web \cite{uscert2012}.
\end{itemize}

%

Existem muitas outras vulnerabilidades de software que são passíveis de exploração por atacantes. Neste sentido, é de suma importância para o desenvolvimento de softwares mais seguros que os Engenheiros de Software construam códigos com qualidade suficiente que os permitam identificar, corrigir e evitar a inserção de vulnerabilidades. Tendo-se o conhecimento de quais são as principais vulnerabilidades existentes, os Engenheiros devem tratar essas vulnerabilidades desde as primeiras fases de design e desenvolvimento código-fonte, seguindo até o fim do ciclo de vida do desenvolvimento do software. Assim como os desenvolvedores programam aplicando ao código princípios de design, devem evoluir o código aplicando princípios de design seguro tais quais os apresentados por outros trabalhos \cite{saltzer1975} \cite{bishop2003} \cite{mcgraw2002} \cite{a1lshammari2009}.

%

Dentre os princípios de segurança que os Engenheiros de Software podem aplicar no design de seus programas, introduziremos aqui os seguintes princípios:

\begin{itemize}
\item \textbf{\emph{Least privilege}}: este princípio sugere que o usuário deve ter somente os direitos necessários para completar suas tarefas \cite{bishop2003}. Em termos de design de classes, significa que o design mais seguro é aquele cujos métodos realizam o menor número de ações possíveis \cite{a1lshammari2009}.
\item \textbf{\emph{Reduce attack surface}}: este princípio tem como objetivo limitar o acesso a dados não permitidos. Pode-se aplicar este princípio reduzindo-se a quantidade de código executáve, com design prezando por menor números métodos de acesso (públicos) e menor número de parâmetros possíveis que possam afetar atributos privados para realização de uma tarefa. Pode-se também buscar eliminar serviços que são usados somente por poucos clientes.
\item \textbf{\emph{Defend in depth}}: este princípio sugere que os mecanismos de defesas devem ser aplicados na maior extensibilidade possível, mesmo que isso gere redundância. O princípio \emph{defend in depth} busca defender o sistema contra qualquer possível ataque através de implementação de métodos ou mecanismos diferentes de tratamento destes ataques. O design em camadas facilita sua implementação, pois permite dividir os métodos de defesa de acordo com as responsabilidades de cada camada. Como ponto negativo, a implementação de vários mecanismos de defesa pode acrescentar complexidade ao software, aumentando riscos de inserção de outras vulnerabilidades e a dificuldade de encontrá-las.
\item \textbf{\emph{Fail securely}}: este princípio está relacionado ao controle das falhas que possam ocorrer na aplicação. As possíveis falhas existentes em um software devem ser exploradas e tratadas para que o software esteja preparado para responder a estas falhas adequadamente, sem gerar alarmes, quebrar a aplicação e principalmente abrir espaços para mais ataques maliciosos. Com a aplicação do princípio \emph{fail securely} tem-se a identificação e tratamento de erros, a inserção de mecanismos de respostas que facilitam a utilização correta do software e que permita que estes comportamentos sejam testados pelos desenvolvedores. Outra consequência positiva da aplicação deste princípio é que o princípio \emph{defend in depth} é apoiado uma vez que a identificação de possíveis erros reforça a modularização e separação dos métodos de tratamento dos mesmos.
\item \textbf{\emph{Economy of mechanism}}: princípio que se refere a manter o código que implementa mecanismos seguros menor e o mais simples possível. Este princípio é de suma importância para o tratamento de vulnerabilidades no desenvolvimento do software, pois a simplicidade é fundamental para que os Engenheiros de Software possam encontrar erros e corrigí-los. A medida que a complexidade aumenta, os módulos inseguros do software tendem a ficarem ocultos e mais difíceis de serem testados. A aplicação de técnicas de programação do XP tais como \emph{refactoring}, \emph{test-driven development} e programação em pares são fundamentais para alcançar os objetivos deste princípio. Este princípio está extremamente ligado ao princípio de design \emph{KISS - Keep It Simple, Stupid!}, pois ambos enfatizam que evitar a complexidade significa evitar problemas \cite{mcgraw2002}
\item \textbf{\emph{Mediate completely}}: princípio que defende que todos os acessos a quaisquer objetos devem ser verificados para garantir se há permissões para realizar tal ação. Se em algum momento for solicitado a leitura de um objeto, o sistema deve verificar se o sujeito tem permissão de leitura. Caso tenha, deve prover somente os recursos necessários para realização das tarefas que interessa a este sujeito. Esta operação deve se repetir todas as vezes que a requisição ao objeto for feita, não somente na primeira vez \cite{bishop2003}.
\item \textbf{\emph{Separation of duties}}: princípio relacionado a separação de interesses dentro dos métodos e mecanismos de segurança do software. A OWASP sugere a separação entre as entidades que aprovação a ação, entidades que realizam a ação e entidades que monitoram a ação \footnote{\url{https://www.owasp.org/index.php/Separation_of_duties}}. Este princípio está diretamente relacionado com os princípios de design orientado à objetos tais como coesão e separação de interesses. Por outro lado, também mantém forte relação com o princípio de design seguro \emph{Economy of mechanism}, pois proporciona código mais limpo e que podem ser mantidos separadamente.
\end{itemize}

%

O estudo sobre conceituação de vulnerabilidades conhecidas, de princípios de design seguro e da revisão bibliográfica nos permite afirmar que o Engenheiro de Software é o principal responsável por manter a segurança de seus projetos. Este profissional deve se preocupar com os problemas de segurança desde os primeiros passos da concepção do código-fonte e design até o desenvolvimento dos últimos testes automatizados. Verifica-se uma forte relação entre princípios de design de software com os princípios de desing seguro, onde a aplicação de ambos podem prover softwares mais robustos, extensíveis e seguros. 

%

Como já mencionado, as decisões de design são fundamentais para a concepção de um software seguro. Khan \& Khan (\citeyear{khan2010}) enfatizam que a complexidade é o maior desafio para desenvolvedores de software ao projetarem um produto de qualidade que cubra ao máximo aspectos de segurança. Os mesmos autores ainda definem que a complexidade de software orientados a objetos está relacionada principalmente a quantidade de parâmetros de design de um objeto e as relações estabelecidas entre os objetos do projeto. Da mesma forma, a complexidade é um dos principais problemas que afetam a qualidade interna do software, dificultando principalmente a manutenção e evolução do software. Mesmo a complexidade sendo uma propriedade da essência do software e não acidental, conforme afirmado por Brook (FAZER CITAÇÃO), é de suma importância que os desenvolvedores cuidem da complexidade de seus códigos, pois estes esforços reduzem os impactos negativos diretos sobre a estrutura interna do software assim como na segurança do mesmo. Para tanto, faz-se necessário a aplicação dos princípios de bom design e de princípios de design seguro através, por exemplo, da prática de \emph{refactorings} e aplicação de padrões de projeto. A seguir são listadas algumas características observáveis no design do software que podem indicar complexidade \cite{khan2010}:

%

\begin{itemize}
\item Grande número de métodos específicos da aplicação de um objeto afeta a reusabilidade.
\item Árvores de herança profundas.
\item A grande quantidade de números de filhos de uma classe.
\item Alto acoplamento entre objetos.
\item Grande número de métodos públicos de um objeto.
\item Baixa coesão de classes. 
\end{itemize}

%

O encapsulamento é uma das principais características de projetos orientados a objetos, sendo fundamental para estabelecer critérios de relação entre as classes de um projeto. Em termos de segurança, esta é outra característica fundamental que deve ser cuidadosamente pensada no design de códigos seguros, pois se relaciona diretamente com os princípios \emph{reduce attack surface} e \emph{mediate completely}.

%

O princípio \emph{reduce attack surface}, como já apresentado, se baseia fundamentalmente na redução da exposição das estruturas do sistema em relação a interações externas. Em termos de design, a diminuição de métodos públicos de classes assim como a diminuição do acesso as informações internas da classe podem ser obtidos a partir de um maior grau de encapsulamento das estruturas que compõem esta classe. Além disso, a dependência de parâmetros externos para realização de operações internas pode expor maiores detalhes dos mecanismos e algoritmos das classes de um projeto, sendo resultado do baixo grau de encapsulamento das operações, além de aumentar riscos de ataques e dificuldades de correção de problemas de segurança. Há várias opções de \emph{refactorings} que o Engenheiro de Software pode aplicar durante o desenvolvimento para adequar o encapsulamento de suas classes de acordos com os objetivos de segurança do princípio \emph{reduce attack surface}: \emph{Encapsulate Collection}\footnote{\url{http://refactoring.com/catalog/encapsulateCollection.html}}; \emph{Encapsulate Field}\footnote{\url{http://refactoring.com/catalog/encapsulateField.html}}; \emph{Hide Method}\footnote{\url{http://refactoring.com/catalog/hideMethod.html}}; \emph{Remove Parameter}\footnote{\url{http://refactoring.com/catalog/removeParameter.html}}; \emph{Encapsulate Field}\footnote{\url{http://refactoring.com/catalog/encapsulateField.html}};

%

O nível de encapsulamento também está extritamente relacionado com o princípio \emph{mediate completely}, uma vez que um baixo grau de encapsulamento provê diferentes formas de interações com o objeto que, segundo este princípio, devem ser verificadas sempre. Quando se tem diversos pontos de interação com um objeto as verificações necessárias para prover uma interação segura devem ser mais complexas para explorar os perigos inerentes a cada um desses tipos de interação. Portanto, restringir acessos à alguns métodos e atributos públicos podem beneficiar a aplicação do princípio de design seguro \emph{mediate completely}.

%

Os cuidados com o nível de encapsulamento das estruturas do software tem outros impactos sobre o design seguro. A maior parte das decisões de design afetam a complexidade do código-fonte, o que também é verdade para decisões relacionadas a encapsulamentos de classes. A redução dos métodos de acesso diminui as opções de interação com um objeto, tornando a API do objeto mais simples, sendo que o mesmo pode ser dito para a diminuição de parâmetros. Complementarmente, o encapsulamento auxilia na redução do acoplamento entre classes, promovendo maior independência entre os módulos do projeto, aumentando a extensibilidade, manutenibilidade e testabilidade do código. A redução do acoplamento entre classes apoia o design seguro, principalmente na detecção e tratamento de vulnerabilidades através de testes e aplicações dos princípios de design seguro cujos impactos são mais facilmente gerenciáveis.

%

%Introduzir métricas
%A medida que o código cresce e se torna mais complexo, faz-se necessário o uso de indicadores que auxiliem a visualização dos impactos de determinadas mudanças no código-fonte, inclusive em termos de possíveis vulnerabilidades que surgiram.  

%




