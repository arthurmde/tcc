\chapter{Design de Software e Métricas}
\label{cap-metrics}

\section{Design de Software}
\label{sec-design-sw}

Escrever Sub-seção

\section{Métricas em Engenharia de Software}
\label{sec-metrics-esw} 
Uma métrica, no âmbito da Engenharia de Software, provê uma forma de medir quantitativamente atributos relacionados as entidades do software e do processo de desenvolvimento. Assim, métricas são importantes ferramentas para avaliação da qualidade do código-fonte produzido e acompanhamento do projeto. Meirelles (\citeyear{meirelles2013metrics}) destaca que, com métricas de software, propõe-se uma melhoria de processo de gestão com identificação, medição e controle dos parâmetros essenciais do software.

%

Tanto a existência de um bom design quanto a de testes automatizados que exercitem as funcionalidades são características desejáveis em projetos de software.	Boa parte das técnicas modernas da Engenharia de Software (CITAR TÉCNICAS) são voltadas para desenvolvimento com design que proporcione simplicidade, manutenibilidade e testabilidade. Mesmo que a maior parte dessas técnicas tenham sido disseminadas a partir do advento dos Métodos Ágeis e do Software Livre, cujo foco central está em atividades relacionadas ao código-fonte, elas são aplicáveis independentemente da metodologia de desenvolvimento utilizada \cite{meirelles2013metrics}. A valorização por softwares que atendam estes parâmetros de qualidade deve-se ao fato de sempre que o Engenheiro de Software está escrevendo novas linhas de código, um tempo significativo é gasto por ele na leitura e entendimento do código existente, muitas vezes desenvolvidos por outros Engenheiros. Martin (\citeyear{martin2008}) destaca que o código-fonte deve ser escrito para ser entendido principalmente por pessoas, e não pela máquina.

%

Neste sentido, o monitoramento da qualidade de código-fonte é fundamental e pode apoiar a utilização de técnicas de desenvolvimento que visam a melhoria contínua do código. Além disso, as métricas de código-fonte são muito importantes para projetos de software, pois estas podem ser utilizadas tanto como ferramenta para gestão do projeto quanto como referência técnica para tomada de decisões sobre o código-fonte.

%

Métricas de monitoramento de código-fonte possuem natureza objetiva e foram inicialmente concebidas para medir o tamanho e a complexidade do software \cite{henry1984kafura}\cite{troy1981zweben}\cite{yau1985zweben}. Além disso, outras métricas surgiram para avaliar softwares que utilizam paradigmas específicos, não sendo aplicáveis a qualquer tipo de software. Por exemplo, métricas orientada a objetos são usadas para avaliar sistemas orientados a objetos \cite{systa2000}. Métricas OO são destinadas, portanto, para avaliar a coesão de classes, as hierarquias de classes existentes, nível de acoplamento entre classes, reuso de código, dentre outras características.

%

Algumas características importantes ajudam a classificar as métricas de código-fonte. Assim, podemos classificá-las como estáticas e dinâmicas. Como o próprio nome diz, métricas estáticas capturam propriedades estáticas dos componentes de software e não necessita que o software seja executado para que seus valores sejam coletados. Por outro lado, métricas dinâmicas refletem características chaves tais como dependência dinâmica entre os componentes em tempo de execução do software.

%

As métricas de software também podem ser classificadas quanto ao método de obtenção. Métricas primitivas podem ser diretamente coletadas refletindo um valor observável de um atributo, sendo raramente interpretadas independentemente. Por outro lado, métricas compostas são obtidas a partir da relação de uma ou mais métricas, derivada, por exemplo, a partir de uma expressão matemática.

%

Entretanto, as definições de métricas adequadas para o acompanhamento do projeto, dimensionamento do software e principalmente para a aferimento da qualidade do código-fonte são tarefas que aumentam a complexidade de adoção de métricas em projetos de software, assim como destacado por Rakić e Budimac (\citeyear{rakic2011budimac}). Isto se deve a diversos fatores: à grande quantidade de métrica existentes; pouca aderência de algumas métricas com a realidade; diversas formas de interpretação de dados; dificuldades de definir parâmetros para comparação; poucos recursos de visualização de dados; coleta de dados não automatizados ou difíceis. Fenton e Pfleeger (\citeyear{fenton1998}) definem características desejáveis de métricas que orientam a escolha das mesmas enquanto outros autores \cite{meirelles2013metrics}\cite{almeida2010} estudam formas de viabilizar a utilização de métricas pelos desenvolvedores em geral. O presente trabalho visa correlacionar métricas de desenvolvimento de software com objetivo de definir configurações para estabelecer cenários que representem o estado da qualidade do software. Desta forma, espera-se reduzir as dificuldades de utilização de métricas de código fonte, tanto para o acompanhamento gerencial quanto para a tomada de decisões de design por desenvolvedores baseada em evidências. Para tanto, nas próximas seções serão apresentados estudos realizados sobre métricas de monitoramento de código-fonte para sistemas orientados à objetos, métricas para avaliação de vulnerabilidades do software e o estado da arte existente sobre a detecção de Bad Smells e Código Limpo a partir de métricas.

%

\section{Métricas Orientadas à Objetos}
\label{sec-metrics-oo}

Escrever Sub-seção

\section{Métricas de Segurança}
\label{sec-metrics-security}

A segurança de software está relacionada com o contínuo processo de manter a confiabilidade, integridade e disponibilidade nas diversas camadas que o compõe, sendo considerado parte dos requisitos não-funcionais do sistema. Independentemente da criticidade do sistema, a segurança em software deve ser tratada com prioridade dentro do ciclo de vida de desenvolvimento do software. Aggarwal e colaboradores \citeyear{aggarwal2002} cita que o custo e esforço gastos na segurança do software são bem altos, podendo chegar a 70\% to esforço total de desenvolvimento e suporte do software.

%

Entretanto, problemas de segurança são recorrentes em diversos tipos de softwares podendo gerar perdas materiais e humanas em diferentes proporções. Mesmo com o grande esforço destinado a aspectos de segurança, tais problemas são difíceis de solucionar, pois a Engenharia de Segurança de Software está em fase intermediária de desenvolvimento (CITAR ONDE FALA ISSO). Gandhi e colaboradores \citeyear{gandhi2013} realça as dificuldades de se detectar vulnerabilidades no estágio operacional do software, pois os problemas de segurança não são endereçados ou suficientemente conhecidos nas fases iniciais do desenvolvimento de software. Como resultado, poucas ações são tomadas por Engenheiros de Software especificamente para manter a segurança no desenvolvimento de novas funcionalidades ou até mesmo na realização de refactorings. Em outras palavras, muitas vezes o Desenvolvedor de software pode estar inserindo vulnerabilidades no código que podem ser exploradas por usuários maliciosos ou, acidentalmente, por usuários comuns. Mesmo os Engenheiros de Software que realizam testes unitários automatizados tendem a não exercitar estas vulnerabilidades, pois no geral testam principalmente as condições de uso padrão do software, enquanto deveriam explorar melhor o comportamento do software à interações indesejadas \cite{vries2006}.

%
%
%
%

Formalmente, uma vulnerabilidade pode ser compreendida como uma instância de uma falha na especificação, desenvolvimento ou configuração do software de tal forma que a sua execução pode violar políticas de segurança, implícita ou explícita \cite{krsul1998}. 

\section{Bad Smells}
\label{sec-bad-smells}

Escrever Sub-seção

\section{Código Limpo}
\label{sec-clean-code}

Escrever Sub-seção



