\chapter{Cenários de Decisões}
\label{cap-scenery}

Neste capítulo será apresentado o conceito de Cenários de Decisões, que visa contemplar um dos objetivos principais da presente monografia: Definição de cenários a partir de estudos teóricos para melhorar a interpretação e tomada de decisão sobre métricas estáticas de código-fonte. No Capítulo (REFERENCIAR CHAPTER) foi apresentada a importância do \emph{design} de software. Além disso, discutimos o papel do Engenheiro de Software no desenvolvimento de códigos com \emph{design} robusto, limpo e seguro, explorando os principais conceitos, problemas, princípios e práticas que este profissional deve conhecer para alcançar este objetivo. 

%

Na seção (CITAR SEÇÂO) do Capítulo (REFERENCIAR CHAPTER) ainda foram introduzidos temas relacionados ao monitoramento de código-fonte, métricas de design de software e métricas de vulnerabilidades. Aferimos que apesar de vários estudos relacionados à utilização de métricas de código-fonte ainda existem muitas dificuldades da adoção prática de métricas de código-fonte em projetos reais de Software. Deve-se enfatizar que métricas não resolvem problemas e sim as pessoas \cite{}. Métricas de software atuam como indicadores para prover informação, entendimento, avaliação, controle e predição para que as pessoas possam realizar escolhas e ações. Neste sentido, no presente capítulo buscamos reduzir a distância entre a medição do código-fonte e tomada de decisões por Engenheiros de Software através da proposta de Cenários de Decisões.

%






%------------------------------------------------------------------------------%

\section{Objetivos}




%------------------------------------------------------------------------------%

\section{Metodologia e Pesquisa}
