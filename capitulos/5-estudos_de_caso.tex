\chapter{Estudo de Caso}
\label{cap-case-study}

%

Neste capítulo será apresentada a utilização dos Cenários de Decisão em projetos reais, com o principal objetivo de demonstrar a reprodução de cenários em ambientes de tomada de decisão. Para tanto, será apresentada a avaliação de três projetos de softwares livres em C++. Além disso, iremos explicar os passos necessários para reproduzir a estrutura de cenários nos dois ambientes de tomadas de decisões abordados nesta monografia: Mezuro e DWing. Assim, será apresentado as principais evoluções e adaptação de cada ferramenta e os detalhes específicos da observação de cada cenário.

\section{Criação dos Cenários do Mezuro}
\label{mezuro-cenarios}

explicar primeiro como o Mezuro foi utilizado para suportar os Cenários, o que foi adaptado e melhorado para melhor suportar cenários, etc.

Depois deve-se mostrar a estrutura de cada cenário dentro da ferramenta (scripts e configuração)

\section{Criação dos Cenários do DWing}
\label{dw-cenarios}

explicar primeiro como o DW foi utilizado para suportar os Cenários, o que foi adaptado e melhorado para melhor suportar cenários, etc.

Depois deve-se mostrar a estrutura de cada cenário dentro da ferramenta (scripts e configuração)

\section{Projetos analisados}
\label{cap-projects}

Nesta monografia definimos a técnica de Cenários de Decisão e propomos um conjunto de cenários para tomada de decisão sobre a segurança do software, seja a partir da utilização de métricas de \emph{design} ou através de métricas de vulnerabilidades. Para apresentar como estes cenários podem ser utilizados, foram analisados três softwares livres. 

No contexto desta monografia, restringiu-se a escolha dos projetos para softwares desenvolvidos em C++ devido as seguintes motivações:

\begin{itemize}
\item Todos os cenários criados se aplicam em projetos escritos em C++.
\item No Capítulo \ref{cap-cenarios}, já foram definidos os valores de referência que podem ser utilizados para C++.
\item Existem uma quantidade expressiva de projetos livres desenvolvidos em C++, que incluem grandes projetos renomados como Chrome, Firefox, MySQL e OpenOffice.
\item Ambas as ferramentas utilizadas já suportam a coleta da maior parte das métricas introduzidas no Capítulo \ref{cap-metrics} para C++.
\end{itemize}

A seguir é feita uma pequena introdução à cada um dos projetos escolhidos.

\subsection{GNU Octave}
\label{section-octave}

O GNU Octave é um projeto desenvolvido em C++ que consiste em uma linguagem interpretada de alto nível destinada à computação matemática distribuída através da licença GNU General Public License \footnote{\url{http://www.gnu.org/copyleft/gpl.html}}. Ele possui uma interface baseada em linha de comando para a solução de problemas matemáticas lineares e não lineares, assim como possibilita a execução de experimentos numéricos. Além disso, também provê um conjunto extensivo de ferramentas para geração de gráficos, visualização e manipulação de dados.

Outras informações específicas sobre o projeto GNU Octave pode ser obtidas através da documentação oficial do projeto, disponível na página \url{https://www.gnu.org/software/octave/}.

Este projeto será analisado tanto através do Mezuro quanto através do DWing.


==== MOSTRAR OS CENÀRIOS EM CADA FERRAMENTA ====


\subsection{Projeto 2}
\label{}

Projeto a ser analisado somente através do Mezuro

\subsection{Projeto 3}
\label{}

Projeto a ser analisado somente através do DWing


