\chapter{Conclusão}
\label{cap-consideracoesFinais}

Métricas são ferramentas importantes que podem ser utilizadas em projetos de software, sendo também muito estudadas na academia e utilizadas na Engenharia de Software Experimental. Entretanto, ainda existem muitas dificuldades inerentes a adoção de métricas nesses projetos. Neste trabalho buscamos apresentar a técnica de Cenários de Decisões como alternativa para monitoramento através de métricas de código-fonte e utilizá-la a partir da evolução e adaptação de duas ferramentas.

%

A técnica de medição Cenário de Decisão, proposta neste trabalho, se apresenta como uma forma alternativa de utilização de métricas em projetos de software cujos benefícios não foram avaliados experimentalmente no contexto desta monografia.  Tomar uma decisão em cima do resultado de uma métrica específica é perigoso, pois uma métrica pode representar várias situações, que podem ser melhor identificadas com a análise de outras métricas. Por isso, a criação de cenários busca identificar situações específicas de código fonte baseado, em sua maioria, em um conjunto de métricas, permitindo assim uma tomada de decisão mais segura. 

%

Observou-se que outros trabalhos já utilizaram conceitos e abordagens semelhantes aos Cenários de Decisões. Sendo assim, outros trabalhos que utilizam ou propoem  métricas de software podem vir a utilizar a estrutura proposta de Cenários de Decisões para diminuir os problemas inerentes ao processo de medição. Entretanto, como discutido durante os Capítulos \ref{cap-metrics-esw} e \ref{cap-cenarios}, os   Cenários de Decisões possuem potencial para serem utilizados principalmente no desenvolvimento de projetos de software para apoiar o monitoramento do código-fonte e o contínuo processo de melhoria da qualidade interna do produto. 

%

Outro objetivo deste trabalho era aplicar a técnica de medição proposta para melhorar o monitoramento de aspectos de segurança do código-fonte. Para isso, a primeira etapa deste trabalho apresentou uma revisão bibliográfica sobre qualidade interna de software, abrangendo aspectos relacionados ao \emph{design} e segurança. 

%

Com esse estudo, averigou-se que métricas de \emph{design} de código podem ter relações com  vulnerabilidades de software, pois foi observado que um código com baixa qualidade e alta complexidade facilitam a inserção de erros e dificultam a identificação de vulnerabilidades já existentes. Além disso, discutiu-se que a aplicação dos princípios de segurança em software envolve decisões de \emph{design}. Dessa forma, algumas métricas de \emph{design} em conjunto podem determinar vulnerabilidades de software, dado que a qualidade interna do código-fonte influencia na inserção de vulnerabilidades e na identificação e remoção destas vulnerabilidades e falhas.

%

Neste sentido, quatro cenários foram propostos para monitorar a segurança de projetos de software a partir de métricas de \emph{design} e outros cinco cenários foram criados para monitorar a segurança do código-fonte a partir de métricas específicas de vulnerabilidades, para os quais foram definidos os valores de refêrencia para softwares desenvolvidos em C++. O primeiro conjunto de cenários pode ser utilizado em diversos contextos, uma vez que as métricas envolvidas são extraídas por vários extratores e não dependem da linguagem de programação do projeto. Por outro lado, o segundo conjunto de cenários são específicos para projetos em C/C++ e tratam de problemas de segurança diretamente ligados a essa linguagem.

%

Observamos ao longo desta monografia que, para o contexto de vulnerabilidade de software, métricas relacionadas ao \emph{design} de código podem ser mais facilmente agrupadas em Cenários de Decisão do que métricas de vulnerabilidades específicas. Isto acontece pois as métricas de vulnerabilidades identificam ocorrências específicas das mesmas, sendo difícil relacionar uma métrica com outra. Além disso, cada uma dessas vulnerabilidades são documentadas por CWE's específicas e, portanto, suas ocorrências e soluções são bem definidas. Entretando, o estudo sobre taxonomias de vulnerabildiades sugeriu uma maneira de agrupar algumas vulnerabilidades específicas em vulnerabilidades mais alto nível. Um exemplo disso é a vulnerabilidade \emph{Buffer Overflow}, que pode ocorrer de diferentes formas, como uso de funções como \emph{gets()}, ou pela má manipulação de \emph{arrays}, etc. Baseado nisso,  buscou aplicar esse tipo de conceito na criação dos cenários de decisão para vulnerabilidades específicas.

%

Por fim, a última contribuição do trabalho consistiu na realização de dois estudos de caso que visaram a utilização de Cenários de Decisões em ferramentas de tomada de decisão. 

O primeiro estudo consistiu na evolução e configuração da plataforma livre de monitoramento de código-fonte Mezuro para suportar a técnica de cenários. Observou-se através desse estudo que a técnica pode ser utilizada com sucesso dentro da plataforma, onde já existem dois projetos, avaliados no contexto desta monografia, que são monitorados a partir dos Cenários de Decisões de Design Seguro. Neste primeiro estudo de caso, observou-se que algumas limitações favorecem a utilização do Mezuro para acompanhar o desenvolvimento e desfavorecem a utilização da plataforma para práticas gerenciais, no contexto de utilização dos cenários.

O segundo estudo consistiu na evolução de um modelo e ambiente de DW para observar projetos de software a partir de Cenários de Decisões. Neste estudo, verificou-se que o modelo apresentado suporta adequadamente o monitoramento através de cenários. Além disso, verificou-se que os diversos recursos do ambiente DWing proporcionam um excelente ecossistema para gerenciamento de projetos, fornecendo diferentes formas de informações visuais, temporais e estruturais sobre a qualidade do software. 

Tanto as evoluções e cenários criados no Mezuro podem ser utilizados pela comunidade para o monitoramento de outros projetos. Da mesma forma, o modelo e ambiente DWing criado é reprodutível e pode ser adaptado à outros contextos para monitoramento de projetos reais.


%===============================================



\section{Trabalhos Futuros}

Quanto aos trabalhos futuros, gostaríamos de estudar e validar a correlação existente entre métricas de \emph{design} com métricas de vulnerabilidades específicas. A idéia seria verificar através de estudos estatísticos a partir de análise de diversos softwares se de fato métricas de \emph{design} tem relação direta com vulnerabilidades específicas e também que se para projetos não críticos, o cuidado com o \emph{design} e a aplicação de princípios de \emph{design} seguros são o suficiente para termos softwares seguros. 

Outra possibilidade de evolução deste trabalho seria também a validação da utilização da técnica Cenário de Decisão. Esse estudo poderia se basear em um estudo de caso real, onde seria verificado há resultados significativos em relação a qualidade de código se for feito o monitoramento baseado em Cenários de Decisões. Este estudo poderia buscar integrar as duas abordagens de monitoramento apresentado nessa monografia (Mezuro e \emph{DWing}) e utilizar das propostas de Medição Rápida e Medição de Projetos apresentado na Figura \ref{fig:agile-design-metrics}.

%

%Atualmente, a evolução do Mezuro está no ponto de finalização do módulo Kalibro Processor. Portanto, pretendemos contribuir para com a evolução do Mezuro como plataforma independente em diferentes níveis. Dada os objetivos tecnológicos e de pequisa expostos no Capítulo \ref{cap-introducao}, essas contribuições devem ser realizadas nos diferentes módulos que compõem a plataforma, contemplando a inserção de novos coletores de métricas sob o Kalibro, assim como melhorias relacionadas as formas de acompanhamento e configuração de métricas. Essas contribuições serão pertinentes para esta monografia, assim como para a evolução do Mezuro, uma vez que as decisões serão compartilhadas com a comunidade de desenvolvedores e as contribuições em código terão impactos importantes e comuns quando se trata de software livre. Portanto, as contribuições irão ajudar para melhor compreensão da utilização do Mezuro e suas funcionalidades. Além disso, algumas das evoluções desejadas e destacadas no Capítulo \ref{cap-introducao} podem se tornar fundamentais para que o Mezuro possa se tornar não só uma ferramenta com viés acadêmico, mas que também contemple diferentes objetivos dentro de projetos de software. Pretendemos adicionar novos coletores de métricas ao Mezuro, o torando ainda mais flexível e aumentando seu potencial de uso.

%

%A construção de um ambiente de DWing também é um dos principais objetivos desta monografia. Assim como o Mezuro, o ambiente de DWing tem como objetivo ser construído para auxiliar no monitoramento de cenários de vulnerabilidade de software em que uma aplicação se encontra e dar suporte a tomada de decisão. Dessa forma, deverá ser desenvolvida todas as tarefas descritas no ciclo de de vida de desenvolvimento de um DWing (seção \ref{sec-lifecycleDw}), desde a modelagem dimensional, processo de ETL e visualização das informações obtidas para a realização de análises. A fonte de entrada de dados será ferramentas de análise estática de código e a periodicidade de coleta será definida durante a modelagem, pois nesse processo que será definido a granularidade das informações.

%

%A fim de comparar as duas soluções para responder a algumas das questões de pesquisa definidas neste trabalho, será elaborado um protocolo de estudo de caso. Este protocolo será projetado durante os dois primeiros meses de trabalho (Julho e Agosto) para que a sua aplicação seja realizada nos dois meses posteriores (Setembro e Outubro).
%
%Durante todo o período de construção da segunda etapa deste trabalho, queremos refinar e identificar outros Cenários de Decisões. O estudo de correlação entre métricas de \emph{design} e vulnerabilidades de software, assim como mais estudos teóricos são fundamentais para evolução e criação desses cenários. Para tanto, projetaremos e realizaremos o experimento que avaliará a correlação estatística entre as métricas estudadas. Por fim, os dois últimos meses serão destinados a análise dos dados e relato de resultados obtidos. 
	
