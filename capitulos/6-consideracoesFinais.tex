\chapter{Considerações Finais}
\label{cap-consideracoesFinais}


(DAR UM APANHADO GERAL DO TRABALHO)


%respondendo questao de pesquisa  4

Como discutido ao longo deste trabalho, métricas de \emph{design} de código tem relação com  vulnerabilidades de software, pois vimos que um código com baixa qualidade e alta complexidade facilitam a inserção de erros e dificultam a identificação de vulnerabilidades já existentes. Dessa forma, algumas métricas de design em conjunto podem determinar vulnerabilidades de software, visto facilitar a inserção de vulnerabilidades e dificultar a identificação destas no código fonte pode ser considerado uma vulnerabilidade. Foi visto também nas seções \ref{sec-metrics-security} e \ref{subsec-security-metrics} que existem vulnerabilidades específicas, que não são aplicáveis a todas as linguagens, que são independentes da qualidade de software. Essas vulnerabilidades específicas são mais sujeitas a ataques, pois são pontos falhos do software que estão sujeito a exploração por esses atacantes. 

%

Dessa forma, podemos responder parcialmente a questão de pesquisa QP4 definida neste trabalho, que diz respeito a saber se métricas de \emph{design} possui correlação com métricas de vulnerabilidade. Métricas de \emph{design} podem especificar vulnerabilidades de software, porém não necessariamente estão diretamente relacionadas a vulnerabilidades específicas. Mas podemos afirmar que um código que segue bons princípios de design pode reduzir vulnerabilidades, pois em um código limpo e de baixa complexidade é mais difícil inserir vulnerabilidades e é mais fácil a sua identificação. Para responder essa pergunta completamente, será criado um protocolo de estudo de correlação entre esses dois grupos de métricas a fim de analisar softwares livres na comunidade e identificar se tal correlação existe de fato.

%

Neste trabalho também foi questionado se métricas isoladas poderiam se relacionar e compor cenários para definição de indicadores mais informativos e completos (QP6). O fato de que a análise de métricas isoladas podem gerar indicadores não tão confiáveis é um assunto discutido em vários trabalhos. Tomar uma decisão em cima do resultado de uma métrica específica é perigoso, pois uma métrica pode representar várias situações, que podem ser melhor identificadas com a análise de outras métricas. Por isso, a criação de cenários busca identificar situações específicas de código fonte baseado em um conjunto de métricas, permitindo assim uma tomada de decisão mais segura. 

%

Observou-se nesta monografia que, para o contexto de vulnerabilidade de software, métricas relacionadas ao \emph{design} de código podem ser agrupadas em cenários mais facilmente em relação a métricas de vulnerabilidades específicas, visto que vulnerabilidades específicas são definidas por CWE's específicas. O projeto CWE faz o catálogo de vulnerabilidades específicas de código, e estas vulnerabilidades específicas já podem ser considerados cenários completos, dado o fato de que estas vulnerabilidades são bem definidas em termos de como ocorrem e como são solucionadas.   

%refatorar texto acima, acho q ficou um pouco repetitivo.





\section{Proposta de trabalho (OU Trabalhos Futuros)}

(inclui descrição do que será feito e cronograma)


%

A construção de um ambiente de DWing também é um dos principais objetivos desta monografia. Assim como o Mezuro, o ambiente de DWing tem como objetivo ser construído para auxiliar no monitoramento de cenários de vulnerabilidade de software em que uma aplicação se encontra e dar suporte a tomada de decisão. Dessa forma, deverá ser desenvolvida todas as tarefas descritas no ciclo de de vida de desenvolvimento de um DWing (seção \ref{sec-lifecycleDw}), desde a modelagem dimensional, processo de ETL e visualização das informações obtidas para a realização de análises. A fonte de entrada de dados será ferramentas de análise estática de código e a periodicidade de coleta será definida durante a modelagem, pois nesse processo que será definido a granularidade das informações.

%

A fim de comparar as duas soluções para responder a algumas das questões de pesquisa definidas neste trabalho, será elaborado um protocolo de estudo de caso baseado no livro (NAO LEMBRO O NOME DO LIVRO, procurar depois). 

	\begin{table}[H]
	\begin{center}
	    \begin{tabular}{ | p{5cm} | c | c |  c |  c |  c |}
	    \hline
	    Atividade & Julho & Agosto & Setembro & Outrubro & Novembro \\ \hline
	    Contribuições com Mezuro & x & x & x & x &  \\ \hline
	    Criar Ambiente DW & x & x & x & x & \\ \hline
	    Contribuições com Analizo & x & x &  &  & \\ \hline
	    Criação e Refinamento de Cenários de Decisões & x & x & x & x & x\\ \hline
	    Criar protocolo de Estudo de Caso de análise das ferramentas & x & x &  &  & \\ \hline
	    Criar protocolo de estudo de correlação entre métricas de código-fonte e de vulnerabilidade & x & x &  &  & \\ \hline
	    Aplicar estudo estatístico de correlação &  &  & x & x & \\ \hline
	    Aplicar Estudo de Caso sobre as ferramentas &  &  & x & x & \\ \hline
	    Análise de dados e resultados &  &  &  & x & x\\ \hline
	    Relatar contribuições &  &  &  & x & x\\ \hline
	    \end{tabular}
	    \caption{Cronograma para o TCC 2}
	    \label{tab:cronograma}
	\end{center}
	\end{table}