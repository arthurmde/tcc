\begin{resumo}[Abstract]
  \begin{otherlanguage*}{english} 
  
The internal quality is a success factor of software projects because it corresponds to the main aspects of the software such as maintainability and security. Software with good internal quality provides more productivity since it supports the creation of more automated tests, as well as it is more understandable, reduces the risk of bugs, and makes the code changes and developments easier to be done. Therefore, the Software Engineer is a major contributor for this success since he should gather a set of skills and knowledge to apply practices, techniques, and tools for creating secure and well design software. Thus, this research covers the main ideas and concepts related to continuous improvement of source code. In this context, in this degree monograph, we highlight the importance of conducting ongoing activities related to design and security throughout the software project, as well as discuss the importance of using static source code metrics to support decision making at managerial level and technical as well. In this regard, we present the concept of Decisions Scenarios that define an abstraction for metrics choice and interpretation, as well as proposals of examples to use scenarios for measuring software security. To support the use of scenarios and metrics in software development, this work also includes a collaboration for the evolution of a source code monitoring platform called Mezuro and building a datawarehousing solution.
  
  \vspace{\onelineskip}
 
  \noindent 
  \textbf{Key-words}: Metrics; Design; Security; Monitoring; DataWarehousing; Decisions Scenarios; Source Code;
  \end{otherlanguage*}
\end{resumo}


