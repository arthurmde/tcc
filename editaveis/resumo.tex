\begin{resumo}

Este trabalho apresenta um estudo e as colaborações na evolução de uma plataforma para monitoramento de códigos-fonte chamada Mezuro. Essa plataforma é desenvolvida através de um projeto de software livre. Em sua concepção, ela foi pensado como um plugin de uma plataforma de redes sociais, o Noosfero.
%
Com sua evolução, ou seja, sucessivas alterações e com o aumento em tamanho e funcionalidades, aumentou-se também a complexidade do Mezuro, assim como a dificuldade em mantê-la. A equipe de desenvolvimento decidiu então por evoluir essa ferramenta para um aplicação independente.
%
Neste trabalho de conclusão de curso discutimos as principais razões e motivações para a evolução dessa plataforma, assim como os impactos em sua arquitetura e nos seus requisitos de qualidade.
%
Apresentamos também um relato das nossas colaborações nesse projeto de software livre.
 \vspace{\onelineskip}
    
 \noindent
 \textbf{Palavras-chaves}: software livre. evolução de software. métricas de código-fonte.
\end{resumo}
